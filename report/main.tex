\documentclass[12pt,
	english,
	a4paper,
	listof=totoc,
	bibliography=totoc,
	parskip=full*
	]{scrreprt}

\usepackage[utf8]{inputenc}
\usepackage[hidelinks]{hyperref}
\usepackage{amsmath}
\usepackage{paralist}
\usepackage{amsfonts}
\usepackage{amssymb}
\usepackage{caption}
\usepackage{pdfpages}
\usepackage{graphicx}
\usepackage{tabularx}
\usepackage{tabu}
\usepackage{enumerate}
\usepackage{enumitem}
\usepackage{adjustbox}
\usepackage[toc,page]{appendix}
\usepackage{subfigure}
\usepackage[left=3.00cm, right=2.00cm, top=1cm, bottom=1.5cm,includeheadfoot]{geometry}
\usepackage{scrpage2}
\pagestyle{scrheadings}
\usepackage{todonotes}
\usepackage{fixltx2e}
\usepackage[toc,nonumberlist,acronym,style=super]{glossaries}
\usepackage{float}
\usepackage{wrapfig}
\usepackage{lscape}
\usepackage{rotating}
\usepackage{blindtext}
\usepackage{booktabs}
\usepackage[backend=bibtex,style=ieee,sorting=nty]{biblatex}
\usepackage{listings}
\usepackage{color}
\usepackage{courier}
\usepackage{tablefootnote}
\usepackage{slashbox}
\usepackage{multirow}
\usepackage{mathtools}
\usepackage{blindtext}
\usepackage[multiple]{footmisc}
\usepackage{multicol}

\renewcommand\appendixtocname{Appendix}
\renewcommand\appendixpagename{Appendix}
\renewcommand\figurename{Figure}
\setcounter{secnumdepth}{3}

\newcommand{\mytitle}{Building Conversational Dialog-Systems using Sequence-To-Sequence Learning}
\newcommand{\myauthor}{Dirk von Grünigen, Martin Weilenmann}

\newcommand{\ra}[1]{\renewcommand{\arraystretch}{#1}}
\newcommand{\quotes}[1]{``#1''}
\newcommand{\fixme}[1]{\todo[linecolor=red,backgroundcolor=red!25,bordercolor=red]{#1}}
\newcolumntype{M}{>{\begin{varwidth}{4cm}}l<{\end{varwidth}}} %M is for Maximal column

\definecolor{dkgreen}{rgb}{0,0.6,0}
\definecolor{gray}{rgb}{0.5,0.5,0.5}
\definecolor{mauve}{rgb}{0.58,0,0.82}

\newcommand\JSONnumbervaluestyle{\color{blue}}
\newcommand\JSONstringvaluestyle{\color{red}}

% switch used as state variable
\newif\ifcolonfoundonthisline

\makeatletter

\lstdefinestyle{json}
{
	showstringspaces    = false,
	keywords            = {false,true},
	alsoletter          = 0123456789.,
	morestring          = [s]{"}{"},
	stringstyle         = \ifcolonfoundonthisline\JSONstringvaluestyle\fi,
	MoreSelectCharTable =%
	\lst@DefSaveDef{`:}\colon@json{\processColon@json},
	basicstyle          = \ttfamily,
	keywordstyle        = \ttfamily\bfseries,
}

% flip the switch if a colon is found in Pmode
\newcommand\processColon@json{%
	\colon@json%
	\ifnum\lst@mode=\lst@Pmode%
	\global\colonfoundonthislinetrue%
	\fi
}

\lst@AddToHook{Output}{%
	\ifcolonfoundonthisline%
	\ifnum\lst@mode=\lst@Pmode%
	\def\lst@thestyle{\JSONnumbervaluestyle}%
	\fi
	\fi
	%override by keyword style if a keyword is detected!
	\lsthk@DetectKeywords% 
}

% reset the switch at the end of line
\lst@AddToHook{EOL}%
{\global\colonfoundonthislinefalse}

\lstset{
	basicstyle=\ttfamily\footnotesize,
	columns=fullflexible,
	showstringspaces=false,
	numbers=left,                   % where to put the line-numbers
	numberstyle=\tiny\color{gray},  % the style that is used for the line-numbers
	stepnumber=1,
	numbersep=5pt,                  % how far the line-numbers are from the code
	backgroundcolor=\color{white},      % choose the background color. You must add \usepackage{color}
	showspaces=false,               % show spaces adding particular underscores
	showstringspaces=false,         % underline spaces within strings
	showtabs=false,                 % show tabs within strings adding particular underscores
	frame=none,                   % adds a frame around the code
	rulecolor=\color{black},        % if not set, the frame-color may be changed on line-breaks within not-black text (e.g. commens (green here))
	tabsize=2,                      % sets default tabsize to 2 spaces
	captionpos=b,                   % sets the caption-position to bottom
	breaklines=true,                % sets automatic line breaking
	breakatwhitespace=false,        % sets if automatic breaks should only happen at whitespace
	title=\lstname,                   % show the filename of files included with \lstinputlisting;
	% also try caption instead of title  
	commentstyle=\color{gray}\upshape,
	xleftmargin=.2\textwidth, xrightmargin=.2\textwidth
}


\lstdefinelanguage{XML}
{
	morestring=[s][\color{mauve}]{"}{"},
	morestring=[s][\color{black}]{>}{<},
	morecomment=[s]{<?}{?>},
	morecomment=[s][\color{dkgreen}]{<!--}{-->},
	stringstyle=\color{black},
	identifierstyle=\color{blue},
	keywordstyle=\color{red},
	morekeywords={xmlns,xsi,noNamespaceSchemaLocation,type,id,x,y,source,target,version,tool,transRef,roleRef,objective,eventually}% list your attributes here
}

\newcommand{\specialcell}[2][c]{%
	\begin{tabular}[#1]{@{}c@{}}#2\end{tabular}}

\bibliography{biblio}
\nocite{*}

% Glossar entries

\newglossaryentry{cell}{name={cell},description={Any instance of any kind of RNN layer is called a \emph{cell}.}}

\newglossaryentry{epoch}
{
  name={Epoche},
  description={Ein Durchlauf durch alle Trainingsdaten während des Trainings eines CNN wird Epoche genannt}
}

\newglossaryentry{filter}
{
  name={Filter},
  description={Grundbausteine eines CNN. Werden verwendet um die Aktivierungen einer Convolutional Schicht zu berechnen}
}

\newglossaryentry{neuron}
{
  name={Neuron},
  description={Grundbausteine eines NN. Bilden eine mathematische Funktion ab}
}

\newglossaryentry{precision}
{
  name={Präzision},
  description={Wert, welcher angibt wieviel Prozent der beurteilten Datensätze korrekt klassifiziert wurden}
}

\newglossaryentry{layer}
{
  name={Schicht},
  description={Sammlung von mehreren Neuronen. Ein NN setzt sich im Normalfall aus mehreren solcher Schichten zusammen}
}

% Acronyms
\newacronym{GPU}{GPU}{Graphical Processing Unit}
\newacronym{NN}{NN}{Neural Network}
\newacronym{RNN}{RNN}{Recurrent Neural Network}
\newacronym{TBTT}{TBTT}{Truncated Backpropagation Through Time}
\newacronym{LSTM}{LSTM}{Long Short-Term Memory Network}


\makeglossaries

\title{\mytitle}
\author{\myauthor}

\begin{document}

\include{title_page}
\includepdf{img/declaration_of_originality.pdf}

% \newgeometry{top=0.75cm,bottom=1cm}
\chapter*{Abstract}
In the following thesis we are going to investigate into building a end-to-end conversational dialog system based on a sequence-to-sequence learning and recurrent neural networks. We use an already built system as the reference architecture. The main topic of the research is to investigate if it is possible to use model, which can be trained on a single GPU, and still exhibit similar performance as much larger models.

The first parts are dedicated to the development of the software system and the datasets and methods used to conduct the experiments.

The analysis of the resulting models is done under several aspects. First, we are going to look into the learning process and try to distinguish differences between using two different datasets for the training. It is shown that the structure and linguistic nature of different datasets has a clear impact on the learning process of such models. Evaluations to asses the performance follow afterwards, which includes tests with a newly proposed metric based on the Sent2Vec library.

The previous analysis reveals several problems our models exhibit. One of the biggest is that such models tend to generate generic responses which leads to deteriorating results. This does not match with our subjective impression that the model indeed become better over time. To find an explanation for this behavior we investigating into the way the language used by this models evolves over time. This analysis shows that language variety increases with training time and the problem with generic responses decrease. We also show that it is not advisable to use small datasets for multiple epochs of the training without any other regularization methods.

A comparison with two other systems, namely the CleverBot and the results from the paper, then follows. This comparison shows that our models can keep up with others as long as the complexity of the dialogs do not get too complicated. The assumed reasons for the inferior results are the smaller size of our models as well as the shorter training time in comparison with other systems.

For the last part, we do an analysis of several technical specifics. This includes short analyzes of using a beam-search decoder, the aid the soft-attention mechanism provides to the models and the clustering of thought vectors. 

We see this thesis and the resulting models as a success, even though the results are not as good as we expected. We were able to provide multiple interesting results, such as the response ``i m a bot'' to the question ``what are you ?''.

\chapter*{Preface}
First and foremost, we want to thank our advisers Dr. Mark Cieliebak, Dr. Stephan Neuhaus and Jan Deriu for their ongoing support and guidance throughout the time we were writing this exploratory thesis. The weekly discussions were always a pleasure and helped us a lot to shape our ideas and goals. We also want to express gratitude to our families, colleagues and friends for their patience, support and encouragement to go on, as it was sometimes hard to keep up the motivation, especially in times were everything seemed to go wrong. Last but not least, we also want to thank the Institut für Angewandte Informationstechnologie from the ZHAW for providing us with the computational resources required to conduct the experiments.


\tableofcontents
\restoregeometry

\chapter{Introduction}
The development of dialog systems or chatbots which are capable of passing the Turing test~\cite{Turing:1950} is a long lived dream among the research community. Since the publication of the Turing test, a lot of efforts have been taken to achieve this goal, but none was so far able to\footnote{http://isturingtestpassed.github.io/}, even though there were some serious contenders. The research area has gained a lot of attention in the past few years due to the technical advancements occurred over the last two decades. One of the big leaps in technology was the inception of machine learning, especially deep learning. The theoretical fundamentals have been available for a long time, but were not applicable due to the lack of computational power and amount of processed data required by such systems. However, changed in the past few years where deep learning had a huge influence on various different fields, for tasks such as image recognition, natural language processing or anomaly detection. They have proven that it is indeed viable to use this technology nowadays because of the exponential growth in available data and computing power.

In this thesis, we will try to develop a conversational dialog system comparable to the one published in the paper \emph{Neural Conversational Model} by Oriol Vinyals and Quoc V. Le~\cite{Vinyals:2015}. Our main motivation for building such a system is twofold: First, their system is an end-to-end system, meaning that the system as a whole can be designed and trained without the obstacle of managing several different subcomponents necessary for performing such a task (i.e. language understanding, interpretation and generation). Our second thought lies in the sheer enthusiasm about the fact that the system presented in the mentioned paper, despite its simplicity, was able to learn to respond meaningful to important questions, such as:

\begin{center}	
	\textbf{Human}: what is the purpose of life ?\\
	\textbf{Machine}: to serve the greater good .
\end{center}

We were blown away when we first saw the machine answering in such a decisive and comprehensible way we would never have imagined possible.

Our goal is to build a system comparable to the system in the paper, but on a much smaller scale. Our aim is to implement a system which is capable of being trained on a single GPU, instead of requiring an entire server with all its RAM and CPUs. For this reason, we investigate if such a model can also be used if scaled-down significantly, namely half of the original size. We then use this scaled down version of this model to train it on two separate datasets: One on the OpenSubtitles~\cite{Lison:2016} and the other on the Reddit Comment Dataset\footnote{https://archive.org/details/2015\_reddit\_comments\_corpus}, specifically using the comments from the subreddits \emph{movies}, \emph{television} and \emph{films}. We have chosen the first dataset in order to evaluate if a smaller model can compete with the results from the mentioned paper. The main idea behind choosing the second dataset is that we want the resulting dialog system to be able to talk about movie-related topics.

The trained models are then analyzed in different dimensions. First, we are exploring the issue of performance metrics and try if a newly proposed metric based on the Sent2Vec~\cite{Pgj:2017} library is sufficient for evaluating semantic similarity in such a setup. We then analyze, how the model learns to talk and try to evaluate which aspects are important for doing so by analyzing the produced language models and compare these models to the language models found in the mentioned datasets. A comparison between our models and the model found in the paper and the \emph{CleverBot}\footnote{http://www.cleverbot.com/} then follows. Finally, our goal is to analyze if beam-search helps to diversify the answers and if and how the soft-attention mechanism~\cite{Bahdanau:2014} can help the models to increase comprehension.
\chapter{Related Work}
\label{related_work}
\blindtext

\chapter{Fundamentals}
\label{chapter:fundamental}
In this first chapter, we lay out the fundamentals used in the rest of this thesis. At the beginning, we will introduce basic definitions such as \emph{utterance}, \emph{sample} and \emph{conversation}. We will then follow up with an introduction into machine learning using a special variant of \emph{Neural Networks} (NN), called \emph{Recurrent Neural Network} (RNN). We show the basic principles behind it and explain problems this RNNs have in practice, and how the can be solved by utilizing other forms of RNNs. We then show how RNNs can be used to build models, called \emph{Sequence-To-Sequence} (seq2seq) models \cite{Sutskever:2014}, which are capable of learning and using a language in a conversational context.

The following introduction only covers a small part of the spectrum of possibilities with regard to the tasks these models can perform. Basically, we are restricting our explanations to the supervised learning use-case and ignore the unsupervised ones, even though they have a wide area of applications (e.g. dimensionality reduction, regression) as shown by !REFERENZ FÜR UNSUPERVISED LEARNING?!. A basic introduction into the principles of neural networks can be found in appendix \ref{fundamentals:neural_networks}.

\section{Definitions}
\paragraph{Utterance} \blindtext
\paragraph{Sample} \blindtext
\paragraph{Conversation} \blindtext

\section{Recurrent Neural Networks}
\emph{Recurrent neural networks} (RNN) are a special variation of the NNs described in appendix \ref{fundamentals:neural_networks}. The main difference between them is, that RNNs have a recurrence built into it, which allows them to adapt to problems which also have a temporal dimension and are dependent on data from differente timesteps to solve. We're also going into the problems such RNNs have due to their recurrence and how this problem can be solved by exploiting another form of RNN, called \emph{Long Short-Term Memory Networks} (LSMT).

\paragraph{RNNs in principle}
One of the main restrictions of vanilla NNs is the following: Assume a task, where the NN is conditioned to output a prediction on tomorrows weather given yesterdays. Now, if one would use a vanilla NN as described in the appendix, the main restriction would be that the NN has to make its prediction solely based on the weather information of the day before. Such a model does not take into account, that weather is not only dependent on the weather of the previous day, but also on the days before. This could be solved by feeding, say, the weather of the last week to the network instead of just the weather of the previous day. But if new scientific evident now shows, that weather is not only dependent on last week, but also on the last month, probably the last year, we quickly get into problems due to the sheer size of a NN performing such tasks, because the input size grows rapidly. Also, such a NN would still be static in the way, that one cannot simply change the time-window used to feed to the network. If one settles with one month, it will always be able predict the weather based on the last month, but not any different time-window, otherwise the NN has to be retrained using another time-window in order to work again as before. RNNs (see figure \ref{fundamentals:rnn:rolled_vanilla}) try to solve this problem by introducing a recurrence into the network, which allows it to exploit informations not only from the current input, but also from inputs of the past. In a more formal way, one could say that this recurrence allows RNNs to ''exhibit dynamic behaviour across the temporal dimension of the input data''.

\begin{figure}[h]
	\label{fundamentals:rnn:rolled_vanilla}
	\centering
	\includegraphics[width=5cm]{img/rnn_rolled}
	\caption{Basic layout of an RNN.\protect\footnotemark}
\end{figure}
\footnotetext{http://colah.github.io/posts/2015-08-Understanding-LSTMs/}

Before explaining how this recurrence can be used to solve the weather prediction problem, let's first show the equations used for the forwared propagation in an RNN. This equations assume, that the inner part of the RNN consists of three distinct layers, called input, hidden and output layer. The internal structure is similar to the one of NNs, as explained in appendix \ref{fundamentals:neural_network}.

\begin{equation}
\begin{split}
o_t  & =  \varphi(\mathbf{w} \cdot \mathbf{x}_t + \mathbf{u} \cdot \mathbf{h}_{t-1}) \\
& = \varphi\bigg(\sum_{i=0}^{n} w_i x_{ti} + \sum_{i=0}^{n} u_i h_{t-1i}\bigg)
\end{split}
\label{fundamentals:rnn:forward_equation:hidden}
\end{equation}


\begin{equation}
\begin{split}
h_t  & =  \varphi(\mathbf{w} \cdot \mathbf{x}_t + \mathbf{u} \cdot \mathbf{h}_{t-1}) \\
     & = \varphi\bigg(\sum_{i=0}^{n} w_i x_{ti} + \sum_{i=0}^{n} u_i h_{t-1i}\bigg)
\end{split}
\label{fundamentals:rnn:forward_equation:output}
\end{equation}

In the equations above, one can see different variables with different indices which we'll explain briefly:

\begin{itemize}[noitemsep]
	\item $x_t$ stands for the input at timestep $t$.
	\item $o_t$ stands for the output of the network at timestep $t$.
	\item $h_t$ stands for the output of the hidden layer before applying the activation function.
	\item $\mathbf{w}$ and $\mathbf{u}$ stand for the weight matrices which are learnt while training.
	\item $\varphi$ is the activation function to be used on the output of the neurons in the hidden layer.
\end{itemize}

\paragraph{Vanishing / Exploding Gradient Problem}
\begin{itemize}
	\item Problem RNNs face in practice
	\item Explain unrolling briefly
	\item Explain gradient descent with TBTT
	\item Show how the identity function on the recurrence can solve this problem.
	\item Reference formal proofs for the problem (with links to dynamical systems).
\end{itemize}

\paragraph{Long Short-Term Memory Networks}
\begin{itemize}
	\item Show basic structure with gates.
	\item Explain that identity function solve the vanishing/exploding gradient problem.
	\item Explain gradient descent with TBTT
	\item Show how the identity function on the recurrence can solve this problem.
	\item Adapt weather prediction problem to LSTM.
\end{itemize}

\section{Sequence-To-Sequence Learning}
\paragraph{Model} 
\begin{itemize}
	\item Explain basic idea behind model.
	\item Show different tasks which can be solved by using such models..
\end{itemize}

\paragraph{Decoding Approaches}
\begin{itemize}
	\item Explain two approaches: Greedy and Beam-Search.
	\item Explain why greedy might not give satisfactory results.
\end{itemize}
\paragraph{Soft-Attention Mechanism}
\begin{itemize}
	\item Draw link to "human" attention mechanism
	\item Show how this works on a more formal level.
\end{itemize}

\section{Technischer Aufbau}
\label{technical_setup}
Im folgenden Abschnitt wird der technische Aufbau erläutert, welcher verwendet wird, um die in Kapitel \ref{sec:Experimente_Resultate} beschriebenen Experimente durchzuführen. Eine Beschreibung zur Verwendung des Systems befindet sich in Anhang \ref{appendix:software_usage}.

\paragraph{Vorarbeiten}
\label{technichal_setup:prework}
Der Grundaufbau der verwendeten Software wurde vom InIT mithilfe von \texttt{keras}\footnote{https://keras.io/} implementiert und zur Durchführung dieser Arbeit zur Verfügung gestellt. Im Rahmen dieses Grundaufbaus wurden die folgenden Funktionalitäten bereits implementiert:

\begin{itemize}[noitemsep]
	\item Implementation des CNN in \texttt{keras} und verwendung von \texttt{theano} \cite{theanoCitShort} als Backend für die \gls{GPU}s.
	\item Implementation von Evaluations-Metriken.
	\item Skripte mit den folgenden Funktionalitäten: Trainieren des CNN, Laden von TSV Dateien, Vorverarbeiten von Word-Embeddings.
\end{itemize}

\paragraph{Anforderungen}
\label{technical_setup:requirements}
Ein zu implementierendes System, mit welchem die Experimente durchgeführt werden können, soll die folgenden Eigenschaften aufweisen:

\begin{itemize}
	\item \textbf{Parametrisierbarkeit}: Dadurch, dass eine grosse Anzahl kleiner Experimente durchgeführt werden muss, soll das System die Möglichkeit bieten, Experimente parametrisiert durchzuführen.
	\item \textbf{Wiederholbarkeit}: Experimente sollen mit einem minimalen Mehraufwand mehrfach durchgeführt werden können.  
	\item \textbf{Übersichtlichkeit}: Resultate der Experimente sollen übersichtlich und einfach zugänglich sein.
	\item \textbf{Auswertbarkeit}: Resultate sollen automatisiert ausgewertet werden können.
\end{itemize}

\paragraph{Funktionalität}
\label{technical_setup:functionality}
Um ein System, welches die oben beschriebenen Anforderungen erfüllt zu erhalten, werden die folgenden Komponenten implementiert:

\begin{itemize}
	\item \textbf{Executor}: Der \emph{Executor} ist zuständig für das Training der CNNs mithilfe von \texttt{keras}. Beim Start akzeptiert er die Konfiguration als Parameter. Das Experiment wird mit dem Laden der benötigten Daten und dem anschliessenden Training des CNN gestartet. Am Ende jeder Epoche wird das aktuelle CNN auf den Validierungsdaten getestet und die konfigurierten Metriken ausgewertet. Diese werden am Ende zusammen mit dem trainierten CNN (Gewichte im HDF5-Format\footnote{https://support.hdfgroup.org/HDF5/}, das CNN Model als JSON) in einen für das Experiment vorgesehenen Ordner gespeichert. Die Metriken werden ebenfalls in dem dafür vorgesehenen Ordner abgespeichert.
	\item \textbf{Config Management}: Experimente werden über Konfigurationen im JSON-Format\footnote{http://www.json.org/} parametrisiert. Über diese Konfiguration können viele wichtige Parameter für die Ausführung festgelegt werden, so zum Beispiel: Anzahl Epochen, Trainings- und Validierungsdaten, Parameter für die k-fold Cross-Validation oder auch bereits trainierte Modelle können geladen werden. Detailierte Erläuterungen zu den einzelnen Parametern können im Anhang \ref{appendix:software_usage} gefunden werden.
	\item \textbf{DataLoader}: Mithilfe des \emph{DataLoader} können Trainings- und Validierungsdaten im TSV\footnote{https://reference.wolfram.com/language/ref/format/TSV.html} Dateiformat geladen werden. Die zu ladenden Daten können dabei aus einer oder mehreren TSV-Dateien stammen. Im Falle, dass mehrere TSV Dateien angegeben werden, kann über die Konfiguration das Verhältnis angegeben werden, in welchem die Daten aus den einzelnen Dateien verschmischt werden sollen.
	\item \textbf{Skripte}: Die Auswertung der einzelnen Experimente geschieht über dafür erstellte Skripte.
	\item \textbf{Weboberfläche}: Auf die Resultate der Experimente kann über eine eigens dafür entwickelte Weboberfläche zugegriffen werden. Ausserdem besteht die Möglichkeit Plots über die Metriken, welche während des Trainings- und Validierungsprozess gesammelt werden, zu erstellen.
	
\end{itemize}
Die oben beschriebenen Komponenten erlauben es, Experimente mittels JSON Konfigurationen zu starten und den gesamten Trainings- und Validierungsprozess mittels Metriken zu überwachen und zu dokumentieren.

\paragraph{Skripte}
\label{technical_setup:scripts}
Für die Durchführung der Experimente wurden diverse Skripte erstellt, um die Handhabung zu vereinfachen und Auswertungen zu ermöglichen. Die Liste der implementierten Scripts umfasst unter anderem die folgenden:

\begin{itemize}[noitemsep]
	\item Erstellen von Plots der Lernkurven und Metriken
	\item Erstellen von Word-Embeddings über einen Text-Corpus
	\item Erstellen von Statistiken zu Trainings- und Validierungsdaten
	\item Vorverarbeitung von Trainingsdaten für die Distant-Phase
	\item Erstellen von Visualisierungen von Word-Embeddings mittels PCA
	\item Diverse Wartungsskripte zur Generierung und Verwaltung von Experimenten
\end{itemize}

\paragraph{Weboberfläche}
\label{technical_setup:webgui}
Um die dritte Anforderung nach Übersichtlichkeit und Auswertbarkeit zu erfüllen, wird eine Weboberfläche umgesetzt, mit welchem die Parameter und Resultate aller durchgeführten Experimente übersichtlich und an einem Ort zur Verfügung gestellt werden. Für die Implementation wird die \texttt{python}\footnote{https://www.python.org/} Bibliothek \texttt{flask}\footnote{http://flask.pocoo.org/} verwendet.

Zur Auswertung der Experimente stehen drei Funktionen zur Verfügung:
\begin{itemize}
	\item Die Oberfläche gewährt Zugriff auf alle JSON Konfigurationen, welche zu einem Experiment gehören. Dazu zählen die Konfiguration selbst, die gespeicherten Trainings- und Validierungsmetriken und das \texttt{keras} Model des CNN.
	\item Mittels der Plotting Funktion können Plots von Trainings- und Validierungsmetriken erstellt werden.
	\item Die gespeicherten Validierungs- und Trainingsmetriken können mithilfe von \texttt{math.js}\footnote{http://mathjs.org/} direkt im Browser ausgewertet werden.
\end{itemize}

\paragraph{Betriebssystem \& Softwarepakete}
\label{technical_setup:software}
Alle Experimente werden mit dem oben beschriebenen Software-System durchgeführt. Auf den beiden verwendeten Computer-Systemen wird als Betriebssystem Ubuntu 16.04 installiert. Dazu werden \texttt{python} in der Version 3.5.2, Nvidia GPU Treiber und \texttt{cuda}\footnote{https://developer.nvidia.com/cuda-toolkit} in der Version 8.0 als Abhängigkeiten von \texttt{theano} und \texttt{keras} installiert.

\paragraph{Hardware}
\label{technichal_setup:hardware}
Zur Durchführung der Experimente werden zwei unterschiedliche Computer verwendet. Im ersten System (S1) ist eine Nvidia GTX970 GPU, einen Intel i7 4950K CPU und 16GB Arbeitsspeicher installiert. Das zweite System besitzt eine Nvidia GTX1070 GPU, einen Intel i7 6700K CPU und ebenfalls 16GB Arbeitsspeicher. Die Unterschiede in der Hardware haben keinen Einfluss auf die Resultate der Experimente, da auf beiden Systemen dasselbe Betriebssystem mit den gleichen Softwarepaketen verwendet wird.
\chapter{Software System}
We had to develop a software system which allows us to conduct experiments for this thesis. The following paragraphs will therefore give a brief explanation on how we've decided to develop the system in order to enable us to conduct experiments with seq2seq models.

\begin{itemize}
	\item Introduce basic idea behind system
	\item Tell that we're using TensorFlow
\end{itemize}
\section{Requirements}
\begin{itemize}
	\item Parameterized system
	\item Easy evaluation of trained models
	\item Allow us to analyze the results
\end{itemize}
\section{Development of the System}
\begin{itemize}
	\item Why we've switch from Keras to Tensorflow
	\item Beginning was really hard, as TF is much more low level than keras
	\item Had huge problems due to different seq2seq APIs with different levels of documentation, bug-freenes and future-support.
	\item Google Seq2Seq Tool -> Inference broken
	\item Settled with legacy API as we know this one is working
\end{itemize}
\section{Model Validation Checks}
\begin{itemize}
	\item Link to previous paragraph and explain why we had to introduce model validation checks.
	\item Explain the overfitting test
	\item Explain the copy test
\end{itemize}

\section{Scripts}
\begin{itemize}
	\item Explain what we've used scripts for
\end{itemize}
\section{Web-UI}
\begin{itemize}
	\item Small description of "talk to model" GUI with screenshots
\end{itemize}
\section{Operating System \& Software Packages}
\begin{itemize}
	\item Declare all dependencies and how the system has to be setup in order to use the software
\end{itemize}
\section{Hardware}
\begin{itemize}
	\item Explain the hardware on the GPU-Cluster, as all experiments were run there
\end{itemize}

\iffalse
\section{Technischer Aufbau}
\label{technical_setup}
Im folgenden Abschnitt wird der technische Aufbau erläutert, welcher verwendet wird, um die in Kapitel \ref{sec:Experimente_Resultate} beschriebenen Experimente durchzuführen. Eine Beschreibung zur Verwendung des Systems befindet sich in Anhang \ref{appendix:software_usage}.

\paragraph{Vorarbeiten}
\label{technichal_setup:prework}
Der Grundaufbau der verwendeten Software wurde vom InIT mithilfe von \texttt{keras}\footnote{https://keras.io/} implementiert und zur Durchführung dieser Arbeit zur Verfügung gestellt. Im Rahmen dieses Grundaufbaus wurden die folgenden Funktionalitäten bereits implementiert:

\begin{itemize}[noitemsep]
	\item Implementation des CNN in \texttt{keras} und verwendung von \texttt{theano} \cite{theanoCitShort} als Backend für die \gls{GPU}s.
	\item Implementation von Evaluations-Metriken.
	\item Skripte mit den folgenden Funktionalitäten: Trainieren des CNN, Laden von TSV Dateien, Vorverarbeiten von Word-Embeddings.
\end{itemize}

\paragraph{Anforderungen}
\label{technical_setup:requirements}
Ein zu implementierendes System, mit welchem die Experimente durchgeführt werden können, soll die folgenden Eigenschaften aufweisen:

\begin{itemize}
	\item \textbf{Parametrisierbarkeit}: Dadurch, dass eine grosse Anzahl kleiner Experimente durchgeführt werden muss, soll das System die Möglichkeit bieten, Experimente parametrisiert durchzuführen.
	\item \textbf{Wiederholbarkeit}: Experimente sollen mit einem minimalen Mehraufwand mehrfach durchgeführt werden können.
	\item \textbf{Übersichtlichkeit}: Resultate der Experimente sollen übersichtlich und einfach zugänglich sein.
	\item \textbf{Auswertbarkeit}: Resultate sollen automatisiert ausgewertet werden können.
\end{itemize}

\paragraph{Funktionalität}
\label{technical_setup:functionality}
Um ein System, welches die oben beschriebenen Anforderungen erfüllt zu erhalten, werden die folgenden Komponenten implementiert:

\begin{itemize}
	\item \textbf{Executor}: Der \emph{Executor} ist zuständig für das Training der CNNs mithilfe von \texttt{keras}. Beim Start akzeptiert er die Konfiguration als Parameter. Das Experiment wird mit dem Laden der benötigten Daten und dem anschliessenden Training des CNN gestartet. Am Ende jeder Epoche wird das aktuelle CNN auf den Validierungsdaten getestet und die konfigurierten Metriken ausgewertet. Diese werden am Ende zusammen mit dem trainierten CNN (Gewichte im HDF5-Format\footnote{https://support.hdfgroup.org/HDF5/}, das CNN Model als JSON) in einen für das Experiment vorgesehenen Ordner gespeichert. Die Metriken werden ebenfalls in dem dafür vorgesehenen Ordner abgespeichert.
	\item \textbf{Config Management}: Experimente werden über Konfigurationen im JSON-Format\footnote{http://www.json.org/} parametrisiert. Über diese Konfiguration können viele wichtige Parameter für die Ausführung festgelegt werden, so zum Beispiel: Anzahl Epochen, Trainings- und Validierungsdaten, Parameter für die k-fold Cross-Validation oder auch bereits trainierte Modelle können geladen werden. Detailierte Erläuterungen zu den einzelnen Parametern können im Anhang \ref{appendix:software_usage} gefunden werden.
	\item \textbf{DataLoader}: Mithilfe des \emph{DataLoader} können Trainings- und Validierungsdaten im TSV\footnote{https://reference.wolfram.com/language/ref/format/TSV.html} Dateiformat geladen werden. Die zu ladenden Daten können dabei aus einer oder mehreren TSV-Dateien stammen. Im Falle, dass mehrere TSV Dateien angegeben werden, kann über die Konfiguration das Verhältnis angegeben werden, in welchem die Daten aus den einzelnen Dateien verschmischt werden sollen.
	\item \textbf{Skripte}: Die Auswertung der einzelnen Experimente geschieht über dafür erstellte Skripte.
	\item \textbf{Weboberfläche}: Auf die Resultate der Experimente kann über eine eigens dafür entwickelte Weboberfläche zugegriffen werden. Ausserdem besteht die Möglichkeit Plots über die Metriken, welche während des Trainings- und Validierungsprozess gesammelt werden, zu erstellen.
	
\end{itemize}
Die oben beschriebenen Komponenten erlauben es, Experimente mittels JSON Konfigurationen zu starten und den gesamten Trainings- und Validierungsprozess mittels Metriken zu überwachen und zu dokumentieren.

\paragraph{Skripte}
\label{technical_setup:scripts}
Für die Durchführung der Experimente wurden diverse Skripte erstellt, um die Handhabung zu vereinfachen und Auswertungen zu ermöglichen. Die Liste der implementierten Scripts umfasst unter anderem die folgenden:

\begin{itemize}[noitemsep]
	\item Erstellen von Plots der Lernkurven und Metriken
	\item Erstellen von Word-Embeddings über einen Text-Corpus
	\item Erstellen von Statistiken zu Trainings- und Validierungsdaten
	\item Vorverarbeitung von Trainingsdaten für die Distant-Phase
	\item Erstellen von Visualisierungen von Word-Embeddings mittels PCA
	\item Diverse Wartungsskripte zur Generierung und Verwaltung von Experimenten
\end{itemize}

\paragraph{Weboberfläche}
\label{technical_setup:webgui}
Um die dritte Anforderung nach Übersichtlichkeit und Auswertbarkeit zu erfüllen, wird eine Weboberfläche umgesetzt, mit welchem die Parameter und Resultate aller durchgeführten Experimente übersichtlich und an einem Ort zur Verfügung gestellt werden. Für die Implementation wird die \texttt{python}\footnote{https://www.python.org/} Bibliothek \texttt{flask}\footnote{http://flask.pocoo.org/} verwendet.

Zur Auswertung der Experimente stehen drei Funktionen zur Verfügung:
\begin{itemize}
	\item Die Oberfläche gewährt Zugriff auf alle JSON Konfigurationen, welche zu einem Experiment gehören. Dazu zählen die Konfiguration selbst, die gespeicherten Trainings- und Validierungsmetriken und das \texttt{keras} Model des CNN.
	\item Mittels der Plotting Funktion können Plots von Trainings- und Validierungsmetriken erstellt werden.
	\item Die gespeicherten Validierungs- und Trainingsmetriken können mithilfe von \texttt{math.js}\footnote{http://mathjs.org/} direkt im Browser ausgewertet werden.
\end{itemize}

\paragraph{Betriebssystem \& Softwarepakete}
\label{technical_setup:software}
Alle Experimente werden mit dem oben beschriebenen Software-System durchgeführt. Auf den beiden verwendeten Computer-Systemen wird als Betriebssystem Ubuntu 16.04 installiert. Dazu werden \texttt{python} in der Version 3.5.2, Nvidia GPU Treiber und \texttt{cuda}\footnote{https://developer.nvidia.com/cuda-toolkit} in der Version 8.0 als Abhängigkeiten von \texttt{theano} und \texttt{keras} installiert.

\paragraph{Hardware}
\label{technichal_setup:hardware}
Zur Durchführung der Experimente werden zwei unterschiedliche Computer verwendet. Im ersten System (S1) ist eine Nvidia GTX970 GPU, einen Intel i7 4950K CPU und 16GB Arbeitsspeicher installiert. Das zweite System besitzt eine Nvidia GTX1070 GPU, einen Intel i7 6700K 
CPU und ebenfalls 16GB Arbeitsspeicher. Die Unterschiede in der Hardware haben keinen Einfluss auf die Resultate der Experimente, da auf beiden Systemen dasselbe Betriebssystem mit den gleichen Softwarepaketen verwendet wird.
\fi
\chapter{Data}\label{chapter:data}
Dieses Kapitel beinhaltet Informationen über die Datensets. Auch erklären wir hier unsere Vorgehensweise, wie wir von den rohen Daten zu unseren Trainingsbeispielen kommen. For the following experiments we used two datasets: The OpenSubtitle dataset includes move subtitles, so this dataset represents spoken language. On the other hand we have the Reddit Submission Corpus dataset which contains written language. We choose those movie related datasets because we focus all subject about movies and films.

\section{Original datasets and preprocessing}
In the table \ref{tbl:data:rawData} you can see some general information about the datasets. Der Reddit Datensatz ist um ein vielfaches grösser, als der OpenSubtitle. Auch scheinen die Einzelnen Aussagen länger zu sein, aufgrund des unterschiedlichen Verhältnisses zwischen der Grösse und der Anzahl Zeilen.
\begin{table}[H]
	\centering
	\small
	\begin{adjustbox}{max width=\textwidth}
		\begin{tabular}{llllll}
			\toprule
			&  \specialcell{\emph{short name}}
			&  \specialcell{\emph{size} \\\textit{[GB]}}
			&  \specialcell{\emph{lines} \\\textit{[million]}}
			&  \specialcell{\emph{data format}}
			&  \specialcell{\emph{source}} \\
			\midrule
			\emph{OpenSubtitle 2016}						&OpenSubtitle	& 93	& 338	& XML	& \cite{lison2016opensubtitles2016}	\\
			\emph{Reddit Submission Corpus 2006-2015} 		&Reddit	&885	& 1'650	& JSON	&  Reddit Comments Corpus \protect\footnotemark \\
			\bottomrule
		\end{tabular}
	\end{adjustbox}
	\caption{Origing and some general information about the Datasets.}
	\label{tbl:data:rawData}
\end{table}
\footnotetext{$https://archive.org/details/2015_reddit_comments_corpus.$}

\subsection{Regular Expressions}
Die Reddit Daten weisen eine Besonderheit auf, wenn der Inhalt [deleted] entspricht, so wird der Datensatz ignoriert, da der Kommentar gelöscht wurde. Was die erlaubten Zeichen betrifft, so gelten für beide Datensätze die gleichen folgenden Regeln:
\begin{itemize}
	\item Gültige Zeichen sind a-z, A-Z, 0-9 und die Satzzeichen .,!?
	\item Wörter welche ungültige Zeichen beinhalten werden entfernt
	\item Zwischen Wörter und Satzzeichen wird immer ein Leerzeichen eingefügt
	\item Grossbuchstaben werden zu Kleinbuchstaben
\end{itemize}
In der Tabelle \ref{tbl:data:regexe} befindet sich eine Beispielaussage vor und nach der Prüfung auf reguläre Ausdrücke.
\begin{table}[H]
	\begin{adjustbox}{max width=\textwidth}
		\centering
		\small
		\begin{tabular}{lll}
			\toprule
			&  \specialcell{\emph{Raw utterance}}
			&  \specialcell{\emph{preprocessed utterance}}\\
			\midrule
			\emph{OpenSubtitle} &\specialcell{Tae Gong Sil was the\\ 'big sun' , and you're 'little sun'.}&\specialcell{tae gong sil was the\\ big sun, and you re little sun .}\\
			\bottomrule
		\end{tabular}
	\end{adjustbox}
	\caption{Abdeckung der Wörter pro Vokabulargrösse und Datensatz.}
	\label{tbl:data:regexe}
\end{table}

\subsection{Preprocessing}
In diesem Kapitel wird erklärt, wie die Trainingsdaten aufgebaut wurden.
\paragraph{OpenSubtitle} Diese Daten sind nach Genre, Jahr und Film sortiert. Wir filterten nicht nach Jahr oder Genre, sondern verwendeten alle Daten für das Training. Jede Datei beinhaltet chronologisch aufsteigend sortiert alle Untertitel eines Filmes im XML Format.
In einem ersten Schritt werden die komprimierten Dateien in den Unterordner gesucht. Anschliessend wird jedes File decompressed und mit Hilfe der XML Library eingelesen. In der Abbildung \ref{fig:data:opus:examp:xml} ist ersichtlich, dass die einzelne Kinder des XML Objekts die Wörter enthalten. Die Wörter werden ausgelesen und entsprechend zu einer Aussage kombiniert. Zwei nachfolgende Aussagen wiederum bilden einen Dialog.
\begin{figure}[h]
	\centering
	\includegraphics[width=7cm]{img/OpenSubtitle_example_xml.PNG}
	\caption{Beispiel XML Struktur einer Aussage im Datensatz OpenSubtitle.}
	\label{fig:data:opus:examp:xml}
\end{figure}
\paragraph{Reddit} Die Reddit Daten liegen soritert nach Jahr und Monat vor. Es gibt ein Datensatz pro Monat, welcher alle Kommentare dieses Monats in JSON Objekten chronologisch aufsteigend beinhaltet. Die Struktur und Attribute eines JSON Objektes sind in der Abbildung \ref{fig:data:reddit:examp:json} ersichtlich. Das JSON Attribut '\emph{body} beinhaltet die eigentliche Aussage.
\begin{figure}[h]
	\centering
	\includegraphics[width=16cm]{img/Reddit_example_json.PNG}
	\caption{Beispiel JSON Struktur einer Aussage im Datensatz Reddit.}
	\label{fig:data:reddit:examp:json}
\end{figure}
Als erstes werden die Daten gefiltert. Je nach Programmparameter nach Jahr und oder Subreddit. Da wir ein Dialog System mit Fokus auf Filme erstellen möchten, filterten wir die Daten entsprechend nach den Subreddit Tags, Movies, Films und Television.\\
Nach dem Filtern der Daten, muss die Ursprüngliche Struktur der Kommentare wiederhergestellt werden. Für diesen Vorgang sind die JSON  Tags, \emph{$Parent_id$}, \emph{$Link_id$} und Name relevant. Auf der obersten Kommentarebene ist die \emph{$Parent_id$} gleich der \emph{$Link_id$}. Wobei beim ersten Vorkommnis pro neue ID ein neuer Beitrag beginnt. Für die Kinder eines Kommentars verhält es sich so, dass deren \emph{$Parent_id$} gleich dem \emph{Name} Tag des Elternkommentar ist.\\
Dialoge werden generiert, indem der Elternknoten jeweils mit allen direkten Kindern kombiniert wird. In der Abbildung \ref{fig:data:reddit:utterance:construction} befindet sich ein theoretisches Beispiel. Es würden folgenden Kombinationen entstehen AB, AC, AD, DE und DF. Zwischen zwei Aussagen befindet sich jeweils unser Trennzeichen ('<<<<<END-CONV>>>>>'), welches während dem Training das Ende einer Konversation bedeutet.
\begin{figure}[h]
	\centering
	\includegraphics[width=10cm]{img/reddit_utterance_construction.PNG}
	\caption{Beispiel Reddit Aufbau der Trainingsdaten.}
	\label{fig:data:reddit:utterance:construction}
\end{figure}
Pro Datensatz erhalten wir nach dem Preprocessing eine Datei, welche pro Zeile eine Aussage enthält.
todo: Clique Bildung in text einbauen? entscheid für subreddit tags Visualisierung der CLique gemäss BigQuery
\section{Vocabulary and Trainingsdata}
\subsection{Vocabulary}
Die Spalte unique amount words in der Tabelle \ref{tbl:data:split:corpus:analyze} beinhaltet die Anzahl verschiedener Wörter pro Datensatz. Der OpenSubtitle besitzt 2.3 Millionen verschiedene Wörter und Reddit 3.1 Millionen. Da wir die Wörter nicht normalisieren können, müssen wir die zur Verfügung stehenden Wörter begrenzen. Ansonsten würde bei der Softmax Berechnung zu wenig Speicher zur Verfügung stehen(Genauer erklären, Referenz auf Kapitel wo erklärt wird). Dementsprechend wird das Vokabular spezifisch für jeden Datensatz generiert.
Dabei werden zuerst alle vorkommenden Wörter gezählt und nach der Häufigkeit absteigend ausgegeben. Wir extrahierten daraus 3 Vokabular, mit den jeweils 25k, 50k und 100k häufigsten Wörter. Die anschliessende Analyse zeigt die Abdeckung des Datensatzes pro Vokabular. Die Ergebnisse sind in der Tabelle \ref{tbl:data:split:corpus:analyze} ersichtlich. Wir sehen hier, dass die Abdeckung über den jeweiligen Dantesatz  mit allen 3 Vokabulargrössen ausreichend wäre.
Wir möchten die Abdeckung nun auf Aussage Ebene untersuchen. Dazu analysierten wir wie oft Wörter pro Aussage nicht erkannt werden. Die Graphen in der Abbildung \ref{fig:data:reddit:vocab:analyze} und \ref{fig:data:opus:vocab:analyze} weisen pro Datensatz ein leicht unterschiedliches Verhalten auf. Der Reddit Datensatz kann mit den 3 Vokabularen deutlich besser abgedeckt werden, als der Opensubtitle. Dies deuted darauf hin, dass die Häufigkeiten der Wörter im Opensubtitle näher beieinander sind, als diejenigen im Reddit Datensatz. Ein Grund hierfür ist der Aufbau der Trainingsdaten, in den Reddit Daten befinden sich einzelne Aussagen mehrfach, jedoch jeweils mit einer unterschiedlichen zweiten Aussage.

Trotz der grossen Anzahl verwendeter Wörter, können bereits mit einem Wörterbuch der Grösser 50k, ~90\% der Wörter abgedeckt werden.

\begin{table}[H]
	\begin{adjustbox}{max width=\textwidth}
		\centering
		\small
		\begin{tabular}{lllllll}
			\toprule
			&  \specialcell{\emph{vocab size}	\\\textit{[thousand]}}
			&  \specialcell{\emph{total amount words}	\\\textit{[thousand]}}
			&  \specialcell{\emph{known words amount}\\\emph{words} \\\textit{[thousand]}}
			&  \specialcell{\emph{known words percent} \\\textit{[\%]}}
			&  \specialcell{\emph{unknown words amount}\\\emph{per utterance} \\\textit{[thousand]}}
			&  \specialcell{\emph{unknown words percent}\\\emph{per utterance} \\\textit{[\%]}}\\
			\midrule
			\emph{OpenSubtitle}	&25		&2362	&2096	&88.73	&266	&11.27\\
								&50		&2362	&2116	&89.57	&246	&10.43\\
								&100	&2362	&2127	&90.03	&236	&9.97\\
			\emph{Reddit}		&25		&1717	&1683	&98.00	&34		&2.00\\
								&50		&1717	&1699	&98.98	&17		&1.02\\
								&100	&1717	&1707	&99.42	&10		&0.58\\
			\bottomrule
		\end{tabular}
	\end{adjustbox}
	\caption{Abdeckung der Wörter pro Vokabulargrösse und Datensatz.}
	\label{tbl:data:split:corpus:analyze}
\end{table}

\begin{figure}[!htb]
	\minipage{0.5\textwidth}
	\includegraphics[width=\linewidth]{img/reddit_vocab_analyze_100k_perc.PNG}
	\centering
	\small
	\text{Reddit 100k}
	\endminipage\hfill
	\minipage{0.5\textwidth}
	\includegraphics[width=\linewidth]{img/opus_vocab_analyze_100k_perc.PNG}
	\centering
	\small
	\text{OpenSubtitle 100k}
	\endminipage\hfill
	\minipage{0.5\textwidth}
	\includegraphics[width=\linewidth]{img/reddit_vocab_analyze_50k_perc.PNG}
	\centering
	\small
	\text{Reddit 50k}
	\endminipage\hfill
	\minipage{0.5\textwidth}
	\includegraphics[width=\linewidth]{img/opus_vocab_analyze_50k_perc.PNG}
	\centering
	\small
	\text{OpenSubtitle 50k}
	\endminipage\hfill
	\minipage{0.5\textwidth}%
	\includegraphics[width=\linewidth]{img/reddit_vocab_analyze_25k_perc.PNG}
	\centering
	\small
	\text{Reddit 25k}
	\endminipage\hfill
	\minipage{0.5\textwidth}%
	\includegraphics[width=\linewidth]{img/opus_vocab_analyze_25k_perc.PNG}
	\centering
	\small
	\text{OpenSubtitle 25k}
	\endminipage
	\caption{Die X-Achse steht für den Prozentsatz fehlender Wörter pro Aussage. Die Y-Achse entspricht der relativen Anzahl zum jeweiligen gesamten Datensatz.}
	\label{fig:data:reddit:vocab:analyze}
\end{figure}
\subsection{Time analyze OpenSUbtitle}
Beim Durchsichten der Daten, empfanden wir die Qualität des OpenSubtitle Datensatzes als eher schlecht. Wir vermuteten die Ursache in der gesprochenen Sprache, da das Bild als wichtiger Informationsträger fehlt. Eine zusätzliche mögliche Begründung wäre es, dass die Bildszene endet und mit einigem zeitlichen Abstand erst wieder gesprochen wird. Somit würden Sätze kombiniert werden, welche nicht den gleichen Kontext besitzen. Deshalb haben wir die zeitlichen Abstände zwischen den Sätzen analysiert. Die Ergebnisse sind in der Grafik \ref{fig:data:analyse:timediff:opus} ersichtlich. Aufgrund der Ergebnisse, dass über 80\% der Aussagen in einem Zeitabstand kleiner gleich 5 Sekunden aufeinander folgen, verwerfen wir vorerst diese These als Ursache. In der Abbildung \ref{fig:data:analyse:timediff:opus} sind die Werte grösser 30\% auffällig hoch. Dies liegt einerseits an den Übergängen zwischen 2 Datensets und weil alle restlichen Werte in dieser Klasse landen. Dies sehen wir aber nicht als Problem an, weil über 80\% der Daten diese Problematik nicht betrifft.
\begin{figure}[h]
	\centering
	\includegraphics[width=15cm]{img/opus_time_analyze.PNG}
	\caption{Relative Verteilung der berechneten Zeiten (Diskret) in ganzen Sekunden zwischen zwei Äusserungen. Wobei die grössten Anteile die folgende 5 Zeitabstände haben: 13.0\% 1 Sekunde, 26.4\% 2 Sekunden, 22.8\% 3 Sekunden, 13.4\% 4 Sekunden und 7\% 5 Sekunden. Die resultierende Abdeckung dieser 5 häufigsten Zeitabstände beträgt 82.6\%.}
	\label{fig:data:analyse:timediff:opus}
\end{figure}
\subsection{N Gram analyze}
Wir untersuchten die Daten zusätzliche nach Häufigkeiten von Bi- und Trigrammen. Diese Vorgang war sehr rechenintensiv und dauerte entsprechend lange. Die Resultate sind pro Datensatz in den Abbildungen \ref{?} und \ref{?} ersichtlich. Das Ziel dieser Analyse ist es, die später generierten Sätze ebenfalls auf Bi- und Trigramme zu untersuchen um eine mögliche Korrelation auf der Ebene Phrase zu finden.
Todo: Einfügen, heat map? und korrelation zwischen n-grams
Bi-Gram Tri-Gram Analyse entweder mit WOlke, oder heat mat, tree map
Vielleicht gibt es noch eine andere Variante um dies darzustellen (besser), wenn möglich mit paper referenz
$http://infosthetics.com/archives/2006/03/subject_tag_news_heat_map.html$
$https://www.quora.com/What-are-alternatives-to-tag-clouds-for-information-visualization$

\subsection{Generating Trainingsets}
Schlussendlich teilten wir die Datensätze auf wobei 97\% als Trainingsdaten, 2\% als Validierungsdaten und 1\% als Testdaten verwendet werden. Die genauen Zahlen befinden sich in der Tabelle \ref{tbl:data:split:corpus}. Die Ursprünglichen Dialoge wurden dabei nur als ganzes aufgeteilt. Die zwei Aussagen des gleichen Dialogs blieben dementsprechend zusammen.
\begin{table}[]
	\centering
	\begin{adjustbox}{max width=\textwidth}
		\centering
		\small
		\begin{tabular}{lllll}
			\toprule
			&  \specialcell{\emph{set}}
			&  \specialcell{\emph{Percent from total} \\\textit{[\%]}}
			&  \specialcell{\emph{size} \\\textit{[MB]}}
			&  \specialcell{\emph{lines} \\\textit{[thousend]}}\\
			\midrule
			\emph{OpenSubtitle}		&train	&97	&9'393	&321'643	\\
									&valid	&1	&97		&3'315	\\
									&test	&2	&194	&6'631	\\
			\emph{Reddit}			&train	&97	&8'455	&75'297	\\
									&valid	&2	&185	&1'552	\\
									&test	&1	&92		&776	\\
			\bottomrule
		\end{tabular}
	\end{adjustbox}
	\caption{In this table you can see the resulting 3 sets per Dataset....}
	\label{tbl:data:split:corpus}
\end{table}

todo: paper relevant? $https://www.researchgate.net/publication/283986649_LSTM-based_Deep_Learning_Models_for_Non-factoid_Answer_Selection$

\chapter{Methods}
\label{methods}
In the following chapter, we are going to elaborate on how we performed the experiments, which hyperparameters where used and how we evaluated the results of the trained models.

\section{Architecture of the Sequence-To-Sequence Model}
In the following paragraphs, we are going to describe the architecture of the model used to conduct our experiments.

\paragraph{Sequence-To-Sequence} In general, we are using the architecture of seq2seq models describe in chapter \ref{fundamentals:seq2seq}. We are using LSTM cells. We restrict ourselves to only use one LSTM cell for the encoder, and another cell for the decoder. This has to do with the fact, that we wanted our cells to be as large as possible to come as close to the size of the cells used in \cite{Vinyals:2015}, as we are trying to replicate the results from there. However, this is not simply possible due to the fact, that in the referenced paper, they used two really large cells with each having a hidden state size of $4096$ hidden units and a vocabulary consisting of $100'000$ words. In the paper, they trained their models on a CPU due to this fact, because such a huge network fits does not fit in the memory of any GPU currently available. As we are seeking to train our models on a single GPU (see chapter \ref{sofware_system:development_history}), we had to shrink the size of our model to the biggest size possible so that it still fits within the 12GB of memory the GPUs we are using have (see chapter \ref{software_system:hardware}). The exact size of the model used in this thesis is described in the subsequent paragraph ``Hyperparameters'' below.

\paragraph{Down-Projection of Hidden State} Because of the problematic with such large RNN cells as described in the preceding paragraph, we implemented a so-called \emph{down-projection} at the end of the decoder cell. This is done similar to the down-projection used in \cite{Vinyals:2015}, with the main difference lying in the motivation why we implemented it. In the paper, they state, that they used it to speed up the training due to the large weights-matrix in the softmax layer at the end. In our case, the down-projection was not just for speeding up the training, but mainly to allow us to use bigger cells than we would be able to without the projection. The implementation of this feature allowed us to grow our model in size by a factor of $2$, from a hidden state size of $1024$ to $2048$, without the need to sacrifice the size of the vocabulary used (see chapter \ref{methods:hyperparameters} for more informations on the used hyperparameters).

\paragraph{Sampled Softmax} To speed up the training of the large softmax layer at the end (consisting of $50'000$ entries), we implemented a \emph{sampled softmax} as described in \cite{Sebastien:2014} which is only applied while training the models. Basically, the idea behind it is, that instead of using the full softmax at each time step in the decoder, we only use a subset of the words from the vocabulary to approximate the softmax layer. This speeds up training dramatically.\todo{Descirbe more!}. On inference time, we then have to use the full softmax layer again to generate the predictions.

\paragraph{Static Unrolling of RNN} Due to the fact, that we are working with a deprecated \texttt{TensorFlow} API (see chapter \ref{sofware_system:development_history}), we have to use static unrolling for our model. Static unrolling works, by defining a fixed size of time steps for the encoder and decoder, and then unrolling the encoder and decoder cell for this number of time steps \emph{before} we are actually computing anything using it. This actually ``removes'' the recurrence from our model and transforms it into kind-of ``feed forward'' model with the major difference being, that the weights are shared between the layers of the unrolled model (as each layer basically represents the same cell).

\begin{figure}
	\label{methods:static_unrolling:unrolled_rnn}
	\centering
	\includegraphics[width=10cm]{img/rnn_unrolled}
	\caption{Image for illustrating the process of unrolling an RNN over a fixed size of time steps.\protect\footnotemark}
\end{figure}
\footnotetext{http://colah.github.io/posts/2015-08-Understanding-LSTMs/}

This implementation forced us to define a maximum number of time steps used for the encoder and decoder beforehand.

\section{Hyperparameters}
\label{methods:hyperparameters}

\paragraph{Model} In summary, for our model we are using the following hyperparameters:

\begin{table}[H]
	\centering
	\ra{1.3}
	\begin{adjustbox}{max width=\textwidth}
		\begin{tabular}{ll}
			\toprule
			Name & Value\\ \midrule
			Number of encoder cells & $1$\\
			Number of decoder cells & $1$\\
			Max. number of time steps in encoder & $30$\\
			Max. number of time steps in decoder & $30$\\
			Hidden state size & $2048$\\
			Projected hidden state size & $1024$\\
			Number of sampled words for softmax & $512$\\
			Size of the softmax layer & $50'000$\\
			\bottomrule
		\end{tabular}
	\end{adjustbox}
	\caption{Hyperparameter, which were used for our seq2seq model.}
	\label{methods:hyperparameters:table}
\end{table}

\paragraph{Optimizer} As the optimizer, we used \emph{AdaGrad} \cite{Duchi:2011} as in \cite{Vinyals:2015} with the learning rate set to $0.01$. We also implemented gradient clipping and set the maximum allowed gradient value to be $10$, as described in \cite{Pascanu:2013}.

\paragraph{Training} We trained our models on the hardware described in chapter \ref{software_system:hardware} with the software packages from \ref{software_system:softwar_packages}.

\section{Evaluation}
\blindtext

\chapter{Analysis of Results}
We are going to analyze the resulting models after training them as specified in Chapter~\ref{methods:training}.

As the first part, we are going to analyze how the training went and take a look at the results of the metric-based evaluation (see Chapter~\ref{methods:evaluation}). Because we identified some problems with this quantitative analysis, we then show that progress was indeed achieved throughout the training and go into detail why the results of the evaluation are as bad as they are. This includes an analysis into the language model to determine why the models often time respond with generic outputs.

The next part is then dedicated to comparing our models with the results from the paper of Vinyals and Le~\cite{Vinyals:2015} and with the CleverBot chatbot.

The subjects of the last part are analysis related beam-search and the soft-attention mechanism.\todo{Evtl. diesen umschreiben/ergänzen}

\section{How Did The Training Go?}
First, let us start by analyzing the evolution of the models with regard to the available performance metrics throughout the time period of the training. Below, in Figures~\ref{results:learning_process:metrics:opensubtitles} and~\ref{results:learning_process:metrics:reddit}, one can see the development of the cross-entropy loss and perplexity values on the training datasets for the two different models.

\paragraph{OpenSubtitles} What is eye-catching when comparing the two is that the OpenSubtitles seems to have much more variance in its performance on the validation set then the Reddit model. This is most probably caused by the fact that the OpenSubtitles dataset is much more noisey than the Reddit, as also noticed by others\cite{Vinyals:2015}. This has to do with the missing information about turn taking, which means it is certainly possible that consecutive utterances in the dataset may be uttered by the same person even though we treat it if it was uttered by two different persons. Also, there is the problematic with the time lags between utterances as analyzed in Chapter~\ref{data:opensubtitles:time_lag_analysis}. In contrast, with the Reddit dataset we always know who uttered a comment and hence can build a dataset which ensures that the dialogs make sense from a structural perspective.

\begin{figure}[H]
	\includegraphics[width=\linewidth]{img/plots/opensubtitles_not_reversed/train_metrics.png}
	\caption{Development of the loss and perplexity on the training and validation set throughout the training of the OpenSubtitles model. One tick on the x-axis is equal to $100$ batches processed.}
	\label{results:learning_process:metrics:opensubtitles}
\end{figure}

\paragraph{Reddit} The learning process of the Reddit model looks find, but it also has a peculiarity, namely the dips in the training loss and perplexity. These dips occur about every $300,000$ to $400,000$ batches. They are also present in the development of the validation loss and perplexity, but are not as apparent as in the training metrics. We cannot explain this behaviour currently. We assume that this comes from the fact...\todo{Maybe we should be able?!} The variance however is much smaller than with the OpenSubtitles model, which strengthens our argument, that a well-structured dataset helps a lot when training such systems as it confuses the model much less.\todo{maybe rewrite this sentence somehow}

\begin{figure}[H]
	\includegraphics[width=\linewidth]{img/plots/reddit/train_metrics.png}
	\caption{Development of the loss and perplexity on the training and validation set throughout the training of the Reddit model. One tick on the x-axis is equal to $100$ batches processed.}
	\label{results:learning_process:metrics:reddit}
\end{figure} 

\paragraph{The Training Seems Successful} From the looks of the plots, it looks like the training went fine for both models, as both of them have degrading loss and perplexity. We also saw differences in how the models have evolved over the time span of the training, especially the dips in the Reddit model, but we cannot conclude anything from this right now. After we have analyzed the training process, we are now focusing on the performance of the models on the test datasets.

\section{Performance on Test Datasets}
After we have seen that the training process looks fine, we are going to asses the performance of these models on our test datasets. Here we use the same metrics as within the training, namely the cross-entropy loss and perplexity values. We have evaluated each model on the respective test dataset for each checkpoint we have created while training (see Chapter~\ref{methods:training}).

\paragraph{Surprising Results} The results on the test set are quite the opposite of the results from the training process (see Figures~\ref{result:test_performance:opensubtitles} and~\ref{result:test_performance:reddit}), where both of the models are getting worse over time. We did not expect that, but nevertheless, we are going to analyze the problem.\todo{Maybe some more clarification on what analyse means?} The results of the OpenSubtitles model seem to vary across the different checkpoints, with the best result having a perplexity of $71.07$ and a loss of $6.15$ and coming from the evaluation with the first checkpoint.\todo{Values for reddit model in text} The best result of the Reddit model is also achieved on the first checkpoint, with all other checkpoints having a worse perplexity.

This result stands in contradiction to what we have expected. Instead of the loss values going up, we would have expected it to go down in the same way as it did on the training and validation datasets. We assume, that this has to do with the cross-entropy loss and hence the perplexity being not the best fit metrics to evaluate such models, especially in a conversational context where the variety of correct answers can be immensely high. For this reason, we proposed a third performance metric, namely the usage of Sent2Vec~\cite{Pgj:2017} embeddings, to measure the similarity between the expected and generated responses. Before we are going to do this analysis, we want to take a look at different samples from both models to show that they indeed improved over time, even thought the test metrics tell a different story.

\begin{figure}[H]
	\minipage{0.5\textwidth}
	\includegraphics[width=\linewidth]{img/plots/opensubtitles_not_reversed/test_metrics_both.png}
	\centering
	\small
	\text{The loss and perplexity.}
	\endminipage\hfill
	\minipage{0.5\textwidth}
	\includegraphics[width=\linewidth]{img/plots/opensubtitles_not_reversed/test_metrics_loss.png}
	\centering
	\small
	\text{Only the loss.}
	\endminipage\hfill
	\caption{The loss and perplexity by running the six different snapshots against the test dataset using the OpenSubtitles model.}
	\label{result:test_performance:opensubtitles}
\end{figure}

\begin{figure}[H]
	\minipage{0.5\textwidth}
	\includegraphics[width=\linewidth]{img/plots/reddit/test_metrics_both.png}
	\centering
	\small
	\text{The loss and perplexity.}
	\endminipage\hfill
	\minipage{0.5\textwidth}
	\includegraphics[width=\linewidth]{img/plots/reddit/test_metrics_loss.png}
	\centering
	\small
	\text{Only the loss.}
	\endminipage\hfill
	\caption{The loss and perplexity by running the six different snapshots against the test dataset using the Reddit model.}
	\label{result:test_performance:reddit}
\end{figure}

\paragraph{Is Progress Achieved?} As the results of evaluating the models on the test dataset has shown, it seems that the models did not make an meaningful progress throughout the training. Even worse, it looks like the models have become worse over time. For this reason, we decided that we would like to evaluate the model by hand and see if the generated responses improve when using the different six snapshots of the models. We opted for four different input sentences, that means two for each model, and generated responses for each of the snapshots. The results can be seen in the Table~\ref{results:example_output:reddit} for the Reddit model and Table~\ref{results:example_output:opensubtitles} for the OpenSubtitles model.

\begin{table}[H]
	\centering
	\begin{adjustbox}{max width=\textwidth}
		\begin{tabular}{ll}
			\toprule
			Input: 	& \texttt{Do you like Joffrey from game of thrones?}\\
			\midrule
			0.5M: 	& \texttt{i m not sure if you re a fan of the movie ,}\\
			& \texttt{but i m not sure if you re a fan of the movie .}\\
			1.0M: 	& \texttt{i m not a fan of the game , but i m not sure if it s a game or not .}\\
			1.5M:	& \texttt{i do nt know what i m talking about .}\\
			2.0M:	& \texttt{i do nt }\\
			2.5M:	& \texttt{i do nt know what i m talking about .}\\
			3.0M:	& \texttt{i do nt know , i just do nt like him .}\\
			\midrule
			Input: 	& \texttt{Which movie can you recommend me?}\\
			\midrule
			0.5M: 	& \texttt{the}\\
			1.0M: 	& \texttt{the movie .}\\
			1.5M:	& \texttt{the one with the}\\
			2.0M:	& \texttt{the dark knight rises}\\
			2.5M:	& \texttt{the prestige}\\
			3.0M:	& \texttt{the movie}\\
			\bottomrule
		\end{tabular}
	\end{adjustbox}
	\caption{Two example dialogs with the Reddit model with one response per snapshot.}
	\label{results:example_output:reddit}
\end{table}
\todo{Format so that all look the same, probably by using tabularx and an X column}

\begin{table}[H]
	\centering
	\begin{adjustbox}{max width=\textwidth}
		\begin{tabular}{ll}
			\toprule
			Input: 	& \texttt{Are you a man or a woman?}\\
			\midrule
			0.5M: 	& \texttt{no , no , no}\\
			1.0M: 	& \texttt{no}\\
			1.5M:	& \texttt{i m not a man}\\
			2.0M:	& \texttt{i m not a woman}\\
			2.5M:	& \texttt{i m a man}\\
			3.0M:	& \texttt{i m not a woman}\\
			\midrule
			Input: 	& \texttt{Why should it not be a good idea to improve you?}\\
			\midrule
			0.5M: 	& \texttt{no}\\
			1.0M: 	& \texttt{i don t know}\\
			1.5M:	& \texttt{because i love you}\\
			2.0M:	& \texttt{because i m a good man}\\
			2.5M:	& \texttt{i m just trying to make a good decision}\\
			3.0M:	& \texttt{i m not a good idea}\\
			\bottomrule
		\end{tabular}
	\end{adjustbox}
	\caption{Two example dialogs with the OpenSubtitles model with one response per snapshot.}
	\label{results:example_output:opensubtitles}
\end{table}

As seen in the examples above, there has indeed been an improvement in the answers, what stands in contradiction to the development of the performance of the models on the test datasets. Our reasonable suspicion from this is, again, that the cross-entropy loss and perplexity are not fit to asses if the responses are meaningful. As we have already described in Chapter~\ref{fundamentals:sent2vec_test}, we were aware of the fact, that our current set of metrics is not satisfying for evaluating if the responses are meaningful, why we decided to use yet another metric, namely Sent2Vec embeddings, which we are going to use in the next section.

\paragraph{Sent2Vec Analysis} As described in Chapter~\ref{fundamentals:sent2vec_test}, we use Sent2Vec embeddings for measuring the semantic similarity between the generated and expected responses on the test dataset. For this purpose, we have used the pretrained models available on the GitHub page of the project\footnote{https://github.com/epfml/sent2vec}. We have decided, that we use both the \emph{Twitter} and \emph{Wikipedia} models for the assessment. The results can be seen in the Figures~\ref{results:sent2vec:opensubtitles:results} and~\ref{results:sent2vec:reddit:results}. What can be directly seen is that both the Reddit and OpenSubtitles models perform better on the pretrained Wikipedia model in comparison to the pretrained twitter model. This probably has to do with the fact that ``compressed'' language used when writing tweets (e.g. ``w/o'' instead of ``without'') and with the coverage of the vocabulary. In conclusion, the results of this analysis are dependent on the Sent2Vec model, as expected. However, both results are pretty bad, with the Reddit results being much better than the OpenSubtitles, about twice as good (see Tables~\ref{results:sent2vec:opensubtitles:results_table} and~\ref{results:sent2vec:reddit:results_table}). The OpenSubtitles model starts with an average similarity of $0.167$ for the first snapshot and climbs up to $0.204$ for the last snapshot. The Reddit model starts with an average value of $0.336$ for the first snapshot and increases to $0.359$ for the last snapshot. This means, that the responses of the Reddit model match the expected responses much better from a semantic perspective as the responses of the OpenSubtitles model do. In summary, the results of both models are highly pretty bad, as the maximum similarity is $1.0$.

\begin{figure}[H]
	\minipage{0.5\textwidth}
	\includegraphics[width=\linewidth]{img/plots/opensubtitles_not_reversed/s2v_wiki_cosine_similarity.png}
	\centering
	\small
	\text{Wikipedia}
	\endminipage\hfill
	\minipage{0.5\textwidth}
	\includegraphics[width=\linewidth]{img/plots/opensubtitles_not_reversed/s2v_twitter_cosine_similarity.png}
	\centering
	\small
	\text{Twitter}
	\endminipage\hfill
	\caption{Results of the evaluation with Sent2Vec on the outputs of the OpenSubtitles models using the pretrained models. The ticks on the x-axis show the different snapshots and the y-axis the average semantic similarity when using Sent2Vec for each snapshot.}
	\label{results:sent2vec:opensubtitles:results}
\end{figure}

\begin{table}[H]
	\centering
	\begin{adjustbox}{max width=\textwidth}
		\begin{tabular}{lcc}
			\toprule
			Snapshot & Avg. Similarity (Wikipedia) & Avg. Similarity (Twitter)\\
			\midrule
			0.5M & $0.16749$ & $0.13827$\\
			1.0M & $0.19111$ & $0.13811$\\
			1.5M & $0.19418$ & $0.14831$\\
			2.0M & $0.19176$ & $0.13840$\\
			2.5M & $0.20118$ & $0.15258$\\
			3.0M & $0.20452$ & $0.16285$\\
			\bottomrule
		\end{tabular}
	\end{adjustbox}
	\caption{The average similarities when applying the Sent2Vec metric on the expected and generated responses from the OpenSubtitles model.}
	\label{results:sent2vec:opensubtitles:results_table}
\end{table}

\begin{figure}[H]
	\minipage{0.5\textwidth}
	\includegraphics[width=\linewidth]{img/plots/reddit/s2v_wiki_cosine_similarity.png}
	\centering
	\small
	\text{Wikipedia}
	\endminipage\hfill
	\minipage{0.5\textwidth}
	\includegraphics[width=\linewidth]{img/plots/reddit/s2v_twitter_cosine_similarity.png}
	\centering
	\small
	\text{Twitter}
	\endminipage\hfill
	\caption{Results of the evaluation with Sent2Vec on the outputs of the Reddit models using the pretrained models. The ticks on the x-axis show the different snapshots and the y-axis the average semantic similarity when using Sent2Vec for each snapshot.}
	\label{results:sent2vec:reddit:results}
\end{figure}
\begin{table}[H]
	\centering
	\begin{adjustbox}{max width=\textwidth}
		\begin{tabular}{lcc}
			\toprule
			Snapshot & Avg. Similarity (Wikipedia) & Avg. Similarity (Twitter)\\
			\midrule
			0.5M & $0.33691$ & $0.30837$\\
			1.0M & $0.33689$ & $0.28694$\\
			1.5M & $0.35340$ & $0.28777$\\
			2.0M & $0.33843$ & $0.30713$\\
			2.5M & $0.35828$ & $0.28908$\\
			3.0M & $0.35956$ & $0.28676$\\
			\bottomrule
		\end{tabular}
	\end{adjustbox}
	\caption{The average similarities when applying the Sent2Vec metric on the expected and generated responses from the Reddit model.}
	\label{results:sent2vec:reddit:results_table}
\end{table}

\paragraph{Generic Response are a Problem} One potential reason for the bad results when using the Sent2Vec metric is that we witnessed both models generating generic responses a lot of the time. To analyze this our first idea was to see, what kind of sentences the models produce with the inputs of the test datasets. We did an analysis on the generated responses and quickly noticed that there are a few sentences, which the models predict a lot of time (see Tables~\ref{results:test_performance:opensubtitles_sample_outputs} and~\ref{results:test_performance:reddit_sample_outputs}).\todo{name percentages of tables}
\\
\begin{table}[H]
	\centering
	\begin{adjustbox}{max width=\textwidth}
		\begin{tabular}{ll}
			\toprule
			Sentence & Frequency\\ \midrule
			\texttt{i m not gon na let you go} & 41853\\
			\texttt{i m not sure i can trust you} & 21263\\
			\texttt{i m not gon na say anything} & 9163\\
			\texttt{i m not gon na let that happen} & 7426\\
			\texttt{i m sorry} & 7235\\
			\texttt{you re not gon na believe this} & 7068\\
			\texttt{you re not gon na believe me} & 6878\\
			\texttt{i m not gon na hurt} you & 4829\\
			\texttt{i m not a fan} & 4468\\
			\texttt{i m not sure} & 4215\\
			\bottomrule
		\end{tabular}
	\end{adjustbox}
	\caption{Top 10 most generated responses with respective occurence frequencies when using the last OpenSubtitles snapshot on the test dataset.}
	\label{results:test_performance:opensubtitles_sample_outputs}
\end{table}\todo{Prozent von Ouptuts als spalte pro satz}\todo{make tables the same width!}

\begin{table}[H]
	\centering
	\begin{adjustbox}{max width=\textwidth}
		\begin{tabular}{ll}
			\toprule
			Sentence & Frequency\\ \midrule
			\texttt{i m not sure if i m being sarcastic or not .} & 17486\\
			\texttt{i think it s a bit of a stretch .} & 13058\\
			\texttt{i m not sure if you re being sarcastic or not .} & 11647\\
			\texttt{i m not sure if i m a <unknown> or not .} & 8307\\
			\texttt{i m not sure if you re joking or not .} & 7932\\
			\texttt{i was thinking the same thing .} & 7579\\
			\texttt{<unknown>} & 6210\\
			\texttt{i m not sure if i m going to watch this or not .} & 4257\\
			\specialcell{\texttt{i m not sure if i m a fan of the show , but i m}\\\texttt{pretty sure that s a <unknown> .}} & 3232\\
			\texttt{i m not sure if i m going to watch it or not .} & 3079\\
			\bottomrule
		\end{tabular}
	\end{adjustbox}
	\caption{Top 10 most generated sentences with respective occurence frequencies when using the last Reddit snapshot on the test dataset.}
	\label{results:test_performance:reddit_sample_outputs}
\end{table}\todo{Prozent von Ouptuts als spalte pro satz}\todo{make tables the same width!}

\paragraph{Does Filtering of Generic Responses Help?} As seen in the both tables above, there are certain responses which are generated a lot of time and are pretty generic and meaningless. Because of that, we thought it would be a good idea to evaluate the models under the Sent2Vec metric one more time, but this time with the top $n$ generic sentences filtered out. We did this with the hope that the generic sentences are the cause of the small average similarities. The results of the analysis with the top $n$ sentences filtered out can be found in Table~\ref{results:sent2vec:opensubtitles:top_n_results_table} and~\ref{results:sent2vec:reddit:top_n_results_table}. For this analysis, we have only used the Wikipedia model as it has shown a better performance for both of our models before.
\\
\begin{table}[H]
	\centering
	\begin{adjustbox}{max width=\textwidth}
		\begin{tabular}{lccc}
			\toprule
			Snapshot & $n = 1$ & $n = 5$ & $n = 10$\\
			\midrule
			0.5M & $0.16679$ & $0.16804$ & $0.16854$\\
			1.0M & $0.19329$ & $0.19394$ & $0.19575$\\
			1.5M & $0.19491$ & $0.19519$ & $0.19539$\\
			2.0M & $0.19215$ & $0.19192$ & $0.19284$\\
			2.5M & $0.20102$ & $0.20127$ & $0.20182$\\
			3.0M & $0.20431$ & $0.20547$ & $0.20568$\\
			\bottomrule
		\end{tabular}
	\end{adjustbox}
	\caption{The average similarities when applying the Sent2Vec metric on the expected and generated responses on the test dataset when filtering out the top $n$ most generated responses the OpenSubtitles model.}
	\label{results:sent2vec:opensubtitles:top_n_results_table}
\end{table}

\begin{table}[H]
	\centering
	\begin{adjustbox}{max width=\textwidth}
		\begin{tabular}{lccc}
			\toprule
			Snapshot & $n = 1$ & $n = 5$ & $n = 10$\\
			\midrule
			0.5M & $0.33772$ & $0.34101$ & $0.34589$\\
			1.0M & $0.34225$ & $0.34238$ & $0.34295$\\
			1.5M & $0.35383$ & $0.35605$ & $0.35564$\\
			2.0M & $0.34008$ & $0.34009$ & $0.34198$\\
			2.5M & $0.35937$ & $0.36142$ & $0.36175$\\
			3.0M & $0.36043$ & $0.35950$ & $0.36313$\\
			\bottomrule
		\end{tabular}
	\end{adjustbox}
	\caption{The average similarities when applying the Sent2Vec metric on the expected and generated responses on the test dataset when filtering out the top $n$ most generated responses for the Reddit model.}
	\label{results:sent2vec:reddit:top_n_results_table}
\end{table}

As seen in the tables above, the filtering of the most used responses does not help a lot when it comes to the Sent2Vec evaluation. We currently cannot say if this problem is just apparent in our use-case or if it this problem is inherent to the metric itself. To get a definitive conclusion, we would have to investigate into using this metric also for other models. \todo{Maybe one more sentence? Maybe say that metric should be applied to ther models (e.g. Google Paper) to see if we are really that bad or if it is inheritent to the metrics itself}

\paragraph{Mixed Feelings about Performance Metrics} As seen in this chapter, the performance metrics used to evaluate the models tell us a mixed story about the resulting models. On one hand, we see that the training went fine and the learning process run as expected. However, when we then test these trained models against the test datasets with the different metrics, it looks like the performance got worse and worse over the time of the training. This is not true in our subjective opinion after ``talking'' to both models for a prolonged period of time. We think the biggest problem for the evaluation are the generic responses both models seem to generate much more than actual answers. To find the cause of this generic responses, we will now try to analyze the language model which both models have learned while training and try to establish a connection between the language model in the datasets and the ones produced by the models.

\section{Development of Language Model}
\label{results:development_language_model}
As said at the end of the previous chapter, in this chapter we are going to investigate into the language models the trained models produce and try to find a substantiation for our claim, that the models generate generic responses much more than.

\subsection{Uni- and Bi-gram Distributions over Time}
As the first step, we are going to compare the uni- and bigram distributions of the training datasets with the distributions produced when evaluating the models with the test datasets. For this purpose we generated unigram and bigram statistics using \texttt{nltk} over the training datasets and outputs generated by the models.

\paragraph{OpenSubtitles} The OpenSubtitles has the most evenly spread distributions when using the first snapshot (i.e. 0.5M). We see a pattern when we look at all distributions of the OpenSubtitles model. In the beginning, it starts with a pretty evenly spread distribution of uni- and bi-grams. As the training goes on, this distributions shift to a more right-leaned when looking at the distributions for the snapshot 1.0M. The distributions then switch back to be more evelny distributed again. This pattern repeats until the end, with one snapshot being evenly distributed whereas the next is then more right-leaned again.

We cannot give a full explanation for this behaviour, but we assume that this has to do with the rather noisey dataset we are using for the OpenSubtitles model. This probably leads to the model being confused, rejecting what it has already learned and trying to fit to the new samples seen. Because of the fact that we cannot explain this behaviour, we are going to analyze the development of the diversity of the uni-/bi-gram and sentences over the course of the training.

The behaviour is the same for the uni-gram distributions, which is the reason we put these visualizations in the Appendix~\ref{APPENDIX UNIGRAM OPUS}.

\begin{figure}[H]
	\minipage{0.5\textwidth}
	\includegraphics[width=\linewidth]{img/plots/opensubtitles_not_reversed/bigram_distribution_comparison_step_500000.pdf}
	\centering
	\small
	\text{Snapshot 0.5M}
	\endminipage\hfill
	\minipage{0.5\textwidth}
	\includegraphics[width=\linewidth]{img/plots/opensubtitles_not_reversed/bigram_distribution_comparison_step_1000000.pdf}
	\centering
	\small
	\text{Snapshot 1.0M}
	\endminipage\hfill
	\minipage{0.5\textwidth}
	\includegraphics[width=\linewidth]{img/plots/opensubtitles_not_reversed/bigram_distribution_comparison_step_1500000.pdf}
	\centering
	\small
	\text{Snapshot 1.5M}
	\endminipage\hfill
	\minipage{0.5\textwidth}
	\includegraphics[width=\linewidth]{img/plots/opensubtitles_not_reversed/bigram_distribution_comparison_step_2000000.pdf}
	\centering
	\small
	\text{Snapshot 2.0M}
	\endminipage\hfill
	\minipage{0.5\textwidth}
	\includegraphics[width=\linewidth]{img/plots/opensubtitles_not_reversed/bigram_distribution_comparison_step_2500000.pdf}
	\centering
	\small
	\text{Snapshot 2.5M}
	\endminipage\hfill
	\minipage{0.5\textwidth}
	\includegraphics[width=\linewidth]{img/plots/opensubtitles_not_reversed/bigram_distribution_comparison_step_3000000.pdf}
	\centering
	\small
	\text{Snapshot 3.0M}
	\endminipage\hfill
	\caption{Comparison of the distributions of the top 100 most used bigrams for the responses of the OpenSubtitles models (orange) when using the test dataset and the distribution within the training data (blue). The distributions are compared for each snapshot available.}
	\label{results:bigram:distributions:opensubtitles}
\end{figure}

\paragraph{Reddit} The distributions for the Reddit model reveal an interesting development (see Figure~\ref{results:bigram:distributions:reddit}). In the beginning, same as with the OpenSubtitles, the distributions are evenly spread. However, as the training goes on, the distributions start to better fit the expected distributions more and more, until it reaches the greatest overlap when using the snapshot 2.0M. From this point on, the distributions start to diverge again. This coincides with our opinion, that the Reddit model is best when using the snapshot 2.0M. This probably has to do with the fact that we are doing more than two epochs over the full dataset. From the snapshot 2.0M on, the model itself starts to become worse, as the diversity and quality of the responses starts to decline. We are going to analyze that assumption below.

The behavior is the same for the uni-gram distributions, which is the reason we put these visualizations in the Appendix~\ref{APPENDIX UNIGRAM REDDIT}.

\begin{figure}[H]
	\minipage{0.5\textwidth}
	\includegraphics[width=\linewidth]{img/plots/reddit/bigram_distribution_comparison_step_500000.pdf}
	\centering
	\small
	\text{Snapshot 0.5M}
	\endminipage\hfill
	\minipage{0.5\textwidth}
	\includegraphics[width=\linewidth]{img/plots/reddit/bigram_distribution_comparison_step_1000000.pdf}
	\centering
	\small
	\text{Snapshot 1.0M}
	\endminipage\hfill
	\minipage{0.5\textwidth}
	\includegraphics[width=\linewidth]{img/plots/reddit/bigram_distribution_comparison_step_1500000.pdf}
	\centering
	\small
	\text{Snapshot 1.5M}
	\endminipage\hfill
	\minipage{0.5\textwidth}
	\includegraphics[width=\linewidth]{img/plots/reddit/bigram_distribution_comparison_step_2000000.pdf}
	\centering
	\small
	\text{Snapshot 2.0M}
	\endminipage\hfill
	\minipage{0.5\textwidth}
	\includegraphics[width=\linewidth]{img/plots/reddit/bigram_distribution_comparison_step_2500000.pdf}
	\centering
	\small
	\text{Snapshot 2.5M}
	\endminipage\hfill
	\minipage{0.5\textwidth}
	\includegraphics[width=\linewidth]{img/plots/reddit/bigram_distribution_comparison_step_3000000.pdf}
	\centering
	\small
	\text{Snapshot 3.0M}
	\endminipage\hfill
	\caption{Comparison of the distributions of the top 100 most used bi-grams for the responses of the Reddit models (orange) when using the test dataset and the distribution within the training data (blue). The distributions are compared for each snapshot available.}
	\label{results:bigram:distributions:reddit}
\end{figure}

\subsection{Language Diversity}
As said in the last Chapter, we are now going to analyze the language diversity of the outputs the two models produce over the different snapshots. For this purpose, we are analyzing the how large the share of the top 10 uni-, bi-gram and sentences is within the generated responses. The results can be seen in the Table~\ref{results:top_10_frequency:reddit} for Reddit and \ref{results:top_10_frequency:OpenSubtitle} for OpenSubtitle. \todo{Zwischen den beiden Corpora gibt es eine besindere Auffälligkeit, die Anzahl benutzter Wörter ist beim reddit viel höher. (wtf *3?? han zahle kontrolliert, aber isch strange).}

\todo{wir könnten schreiben, dass die absoluten Zahlen keine Tendenz anzeigen (das sehen wir schon in der Tabelle), die Werte nehmen absolut gesehen nicht ab, es ist für uns kein Muster erkennbar. Deshalb haben wir Tabelle mit dem prozentualen Anteil ergänzt. Die Werte sind nun nicht mehr ganz so chaotisch, dies hängt damit zusammen, dass eben auch die Anzahl verwendeter uni-gramme und bi-gramme pro Snapshot sich verändert. Dies ist ein Indiz dafür, dass die Sätze länger werden und somit semantisch anspruchsvoller?!

Um diese Enwicklung pro Modell besser zu sehen und auch die Korrelation innerhalb eines Modells zwischen uni-gramm/bi-Gramm und sentence, haben wir die prozentuale Werte geplotet?!}
\todo{weiss nicht wohin damit, aber wollen wir noch erklären, was es bedeuted, wenn die Kurce sinkt? also interpretiert z.B.:Die Werte sind so zu interpretieren, dass wenn die Werte sinken bedeutet dies, dass der Anteil der Top 10 Sätze/uni-gramm/bi-gram an der Gesamtmenge sinkt, was wir als positive Entwicklung werten, weil vielfältiger? Würde nur die Satzkurve sinken, nicht aber die Bi-Gramm Kurve, dann würde es eben auf Stufe bi-gramm nicht vielfältiger werden. Da wir aber auf allen 3 Stufen diese Tendenz sehen, wird es auf allen 3 Stufen vielfältiger?}

\begin{table}[H]
	\centering
	\begin{adjustbox}{max width=\textwidth}
		\begin{tabular}{llllllllll}
			\toprule
			& Words &&&Bi-Gramm&&&Sentences&&\\
			Snapshot & Top Count & All Count& Top freq. \%&  Top Count& All Count& Top freq. \%&  Top Count& All Count& Top freq. \%\\
			\midrule
			0.5M & 2,712,157	 & 4,484,679	 & 60.48\%	&2,096,681	&4,483,620	&46.76\%	&86,408	&249,984	&34.57\%\\
			1.0M & 2,576,291	 & 4,294,209	 & 60.00\%	&1,941,620	&4,290,818	&45.25\%	&71,050	&249,984	&28.42\%\\
			1.5M & 1,918,226	 & 3,668,416	 & 52.29\%	&1,256,740	&3,663,402	&34.31\%	&46,590	&249,984	&18.64\%\\
			2.0M & 2,507,930	 & 4,577,892	 & 54.78\%	&1,665,799	&4,567,229	&36.47\%	&29,544	&249,984	&11.82\%\\
			2.5M & 1,623,645	 & 3,148,834	 & 51.56\%	&1,045,565	&3,134,900	&33.35\%	&73,475	&249,984	&29.39\%\\
			3.0M & 2,108,447	 & 3,614,679	 & 69.38\%	&1,449,193	&3,599,584	&40.26\%	&82,797	&249,984	&33.12\%\\
			\bottomrule
		\end{tabular}
	\end{adjustbox}
	\caption{Top 10 Uni-Gramms, Bi-Gramms and Sentences summed up their frequency and calculated their percentage propotion per Reddit snapshot.}
	\label{results:top_10_frequency:reddit}
\end{table}

\begin{table}[H]
	\centering
	\begin{adjustbox}{max width=\textwidth}
		\begin{tabular}{llllllllll}
			\toprule
			& Words &&&Bi-Gramm&&&Sentences&&\\
			Snapshot & Top Count & All Count& Top freq. \%&  Top Count& All Count& Top freq. \%&  Top Count& All Count& Top freq. \%\\
			\midrule
			0.5M & 516,756	 & 1,035,956	& 49.88\%	&306,757	&1,033,319	&29.69\%	&102,770	&249,984	&41.11\%\\
			1.0M & 639,494	 & 860,951		& 74.28\%	&504,946	&858,145	&58.84\%	&188,748	&249,984	&75.50\%\\
			1.5M & 624,243	 & 1,249,230	& 49.97\%	&378,566	&1,233,467	&30.70\%	&108,312	&249,984	&43.33\%\\
			2.0M & 589,721	 & 814,101		& 72.44\%	&475,120	&807,087	&58.87\%	&181,585	&249,984	&72.64\%\\
			2.5M & 651,779	 & 1,131,640	& 57.60\%	&395,060	&1,117,436	&35.35\%	&92,273		&249,984	&36.91\%\\
			3.0M & 991,879	 & 1,470,695	& 40.10\%	&718,034	&1,459,490	&49.20\%	&105,245	&249,984	&42.10\%\\
			\bottomrule
		\end{tabular}
	\end{adjustbox}
	\caption{Top 10 Uni-Gramms, Bi-Gramms and Sentences summed up their frequency and calculated their percentage propotion per Reddit snapshot.}
	\label{results:top_10_frequency:OpenSubtitle}
\end{table}

\todo{Absatz könnte man dan streichen }
We cannot see a tendency when visualizing the absolute occurrence frequencies of the top 10 uni-, bi-gram and sentences (see Figure~\ref{FIGURE_ABSOLUTE}). For this reason, we started to do the same analysis, but this time with relative values instead of absolute (see Figure~\ref{FIGURE_PERC}). Here we can see the same pattern emerging again as in the last chapter.

\paragraph{OpenSubtitles} The plot for the OpenSubtitles (see Figure~\ref{results:language_model:diversity:opensubtitles}) shows the same patterns as with the distributions seen before. Throughout the course of the training, it looks like that the diversity of the language model wobbles up and down. Our assumption for this flattering development is the same as with the distributions, we think that the noisey training datasets might be responsible for that. But, if we take a close look at the numbers, we see that the peaks are getting smaller and smaller. For example, the share of the top 10 sentences and words for the 2.0M model is about $2\%$ to $3\%$ lower than with the 1.0M model. The difference, when looking at the numbers for the bi-grams is much smaller, only about $0.02\%$. Nevertheless, in total, we can see a tendency that the model starts to use a more diverse language model as longer as the training goes.

\begin{figure}[H]
	\includegraphics[width=\linewidth]{img/plots/opensubtitles_not_reversed/diversity_perc_plot.png}
	\caption{Development how much percentage of all uni-, bi-grams and sentences are covered by the top 10 most used of each category.}
	\label{results:language_model:diversity:opensubtitles}
\end{figure}\todo{Caption!}

\paragraph{Reddit} The plot for the Reddit model (see Figure~\ref{results:language_model:diversity:reddit}) reveals, that the language model gets more diverse throughout the course of the training until the minimum is reached when using the 2.0M model. From this point on, it starts to become less and less diverse for the rest of the training. This coincides with our opinion, that the model is best when using the 2.0M snapshot. Our assumption for this behaviour is, that we are running into problems due to the fact, that between the snapshots 1.5M and 2.0M we are starting the second epoch over the training dataset. This could lead to some kind of overfitting effect, which worsens the results and diversity of the language used by the model.

\begin{figure}[H]
	\includegraphics[width=\linewidth]{img/plots/reddit/diversity_perc_plot.png}
	\caption{Development how much percentage of all uni-, bi-grams and sentences are covered by the top 10 most used of each category.}
	\label{results:language_model:diversity:reddit}
\end{figure}\todo{Caption!}

\subsection{Results}
Our analysis has shown, that the models do indeed produce a large amount of generic sentences, but this problem weakens over the time of the training, especially for the OpenSubtitles model. The second problem we could identify is, that iterating over the same dataset multiple times, does not seem to be advisable. Using a regularization technique, such as Dropout, would probably help in the case of a small dataset.

%Wegen generischen Sätze, wollten wir sparchmodell analysieren
%zuerst haben wir Verteilungen der Bi-Gramme und Uni-Gramme verglichen mit gleichen bi/unigrammen Verteilung aus dem ganzen set. Nähert sich an, wird aber eher rechtslastig, auffällig, wenig genutzte bi/unigramme werden vernachlässigt.
%Anschliessend Sprachvielfalt analysiert, damit konnten wir unseren beim testen bemerkten Lernfortschritt im Opensubtitle MOdell erstmals mit zahlen unterlegen. Wobei, die Sprachvielfalt eher auf Stufe Satz/Wort sich verbesserte, auf Stufe bigramm sehen wir diese entwicklung nocht nicht??? Und es zeigt sich, dass der Anteil der generischen Antworten tendenziell sinkt.
%Bei Reddit konnten wir ebenfalls unseren visuellenen Eindruck bestätigen, das Modell 2.0M war für uns das Beste, und dies wird durch die Grafik \ref{label} bestätigt. Sobald mehrmals mit gleichen Daten trainiert wird, werden resultate subjektiv schlechter, dies bestätigt auch unsere Grafik, dass ab diesem Punkt die Sätze wieder einfältiger werden.
%Wir konnten die Ursache für die genereischen Sätze zwar nicht herausfinden, jedoch konnten wir aufzeigen, dass sich die Vielfält mit fortlaufendem Training langsam aber stetig verbessert.
%
%Die Reddit Kurve \ref{} hingegen sinkt kontinuierlich bis Snapshot 1.5M /2.0 und steigt dann langsam wieder an. Dies deckt sich exakt mit unserem Subjektiven Gefühl, mit welchen Snapshots wir die besten responses erhalten. Zusätzlich deckt es sich mit der Anzahl, wie oft wir mit den gleichen Daten trainieren. Kurz nach dem 2. Durchlauf, beginnt die Kurve zu steigen, was ja bedeuted, dass der Anteil der Top 10 Artefakte an der Gesamtmenge zuniemt. Das Modell verliert seine Vielfalt.
%Wichtig, die einzige Konstante ist die maximale Anzahl Sätze. Somit ist klar, dass wenn die Anzahl benutzter Bi-Gramme/Uni-Gramme steigt, längere Sätze gebildet werden. Diese müssen aber nicht unbedingt besser sein, da es einfach x-mal das gleiche Bi-Gramm/Uni-Gramm sein könnte, ist es aber eben nicht, wie wir an Kurven erkennen.
%
%Wir möchten herausfinden, ob mit Fortlaufendem Training die Sprachvielfalt der responses zunimmt. Für diese Auswertung berücksichtigen wir die top 10 Uni-Gramm, Bi-Gramme und Sätze pro Snapshot und Modell (siehe Anhang \ref{}) und zählen deren Häufigkeit. Die Resultate können Sie aus der Tabelle \ref{} entnehmen, wir haben die Tabelle noch mit der maximalen Anzahl Uni-Gramme, Bi-Gramme und Sätze pro Snapshot ergänzt. Anhand er absoluten Zahlen kann man keine Tendenz angeben, deshalb haben wir den Prozentualen Anteil der Top 10 Artefakte an der Gesamtmenge pro Snapshot berechnet. Diese Werte befinden sich in den Spalten Prozentual.
%Um die Entwicklung zwischen den Snapshot besser zu erkennen, erstellten wir mit den relativen Werten ein Diagramm pro Corpus. Die Kurve ist so zu interpretieren, dass wenn die Werte sinken bedeutet dies, dass der Anteil der Top 10 Artefakte an der Gesamtmenge sinkt. Wir sehen dabei deutliche Unterschiede pro Corpus.
%\paragraph{OpenSubtitles}Die OpenSubtitle Kurve \ref{} zeigt insgesamt eine leichte Verbesserung jedoch mit lokalen Verschlechterungen. Die Spitze beim Snapshot 2.0M ist bei den Sätzen und Wörtern um 2-3 \% tiefer, als diejenige bei 1.0M. Bei den Bi-Grams ist der Unterschied eher klein mit 0.02 \%. Die Ursache für diese Zackenkurve(?) vermuten wir in den Daten. Diese sind wie bereits erwähtn noisy. In der Tendenz sehen wir jedoch ein Gefälle.
%
%\paragraph{Reddit} Die Reddit Kurve \ref{} hingegen sinkt kontinuierlich bis Snapshot 1.5M /2.0 und steigt dann langsam wieder an. Dies deckt sich exakt mit unserem Subjektiven Gefühl, mit welchen Snapshots wir die besten responses erhalten. Zusätzlich deckt es sich mit der Anzahl, wie oft wir mit den gleichen Daten trainieren. Kurz nach dem 2. Durchlauf, beginnt die Kurve zu steigen, was ja bedeuted, dass der Anteil der Top 10 Artefakte an der Gesamtmenge zuniemt. Das Modell verliert seine Vielfalt.
%Wichtig, die einzige Konstante ist die maximale Anzahl Sätze. Somit ist klar, dass wenn die Anzahl benutzter Bi-Gramme/Uni-Gramme steigt, längere Sätze gebildet werden. Diese müssen aber nicht unbedingt besser sein, da es einfach x-mal das gleiche Bi-Gramm/Uni-Gramm sein könnte, ist es aber eben nicht, wie wir an Kurven erkennen.
%
%\paragraph{Fazit}
%Wegen generischen Sätze, wollten wir sparchmodell analysieren
%zuerst haben wir Verteilungen der Bi-Gramme und Uni-Gramme verglichen mit gleichen bi/unigrammen Verteilung aus dem ganzen set. Nähert sich an, wird aber eher rechtslastig, auffällig, wenig genutzte bi/unigramme werden vernachlässigt.
%Anschliessend Sprachvielfalt analysiert, damit konnten wir unseren beim testen bemerkten Lernfortschritt im Opensubtitle MOdell erstmals mit zahlen unterlegen. Wobei, die Sprachvielfalt eher auf Stufe Satz/Wort sich verbesserte, auf Stufe bigramm sehen wir diese entwicklung nocht nicht??? Und es zeigt sich, dass der Anteil der generischen Antworten tendenziell sinkt.
%Bei Reddit konnten wir ebenfalls unseren visuellenen Eindruck bestätigen, das Modell 2.0M war für uns das Beste, und dies wird durch die Grafik \ref{label} bestätigt. Sobald mehrmals mit gleichen Daten trainiert wird, werden resultate subjektiv schlechter, dies bestätigt auch unsere Grafik, dass ab diesem Punkt die Sätze wieder einfältiger werden.
%Wir konnten die Ursache für die genereischen Sätze zwar nicht herausfinden, jedoch konnten wir aufzeigen, dass sich die Vielfält mit fortlaufendem Training langsam aber stetig verbessert.

\section{Comparison with Other Models}
As the next part, we want to compare the results of our model with two others we use as a reference: The CleverBot\footnote{http://www.cleverbot.com/} chatbot and the results from the paper ``Neural Conversational Model''~\cite{Vinyals:2015}, whose model architecture serves as the base of our model.\todo{name differences?} We test the quality of the responses from our models when confronted with utterances of different type and difficulty. We do this in the hope that we can reveal where the strengths and weaknesses of our models lie.

\subsection{Cleverbot}
\label{results:comparison:cleverbot}
We start with the comparison with the CleverBot. Keep in mind, that the CleverBot is a non-deterministic chatbot which means that the responses might be different if it is fed with the same inputs but a different point in time. The responses listed below were generated at the first of June between 12am and 15am.

\paragraph{Simple Utterance} We start by using the two simple utterances ``How are you?'' and ``Hi there, i m john!''. As one can see in the Tables~\ref{results:example1_output:OpenSubtitle_compared} and~\ref{results:example1_output:Reddit_compared}, both our OpenSubtitles model and the CleverBot answered this question with satisfying responses. The responses of the Reddit model to the utterance ``Hi there, i m john!'' are pretty bad as none of them is really useful. The response of the CleverBot to this utterance makes more sense in this case.
\\
\begin{table}[H]
	\centering
	\begin{adjustbox}{max width=\textwidth}
		\begin{tabular}{ll}
			\toprule
			Input: 	& \texttt{Hi, how are you?}\\
			\midrule
			0.5M: 	& \texttt{how are you ?}\\
			1.0M: 	& \texttt{i m fine}\\
			1.5M:	& \texttt{i m good}\\
			2.0M:	& \texttt{good}\\
			2.5M:	& \texttt{good}\\
			3.0M:	& \texttt{good}\\
			Cleverbot:	& \texttt{I'm good. How are you?}\\
			\bottomrule
		\end{tabular}
	\end{adjustbox}
\caption{Responses from the OpenSubtitles model for each snapshot, compared to the CleverBot response. The input utterance is ``How are you?''.}
\label{results:example1_output:OpenSubtitle_compared}

\end{table}
\begin{table}[H]
	\centering
	\begin{adjustbox}{max width=\textwidth}
		\begin{tabular}{ll}
			\toprule
			Input: 	& \texttt{Hi there, i m john!}\\
			\midrule
			0.5M:	& \texttt{<unknown>}\\
			1.0M: 	& \texttt{i m not a fan of the <unknown> , but i m not sure if you re a <unknown> .}\\
			1.5M: 	& \texttt{i m not the only one who thought this was a good idea .}\\
			2.0M:	& \texttt{i m a <unknown> !}\\
			2.5M:	& \texttt{i m not your buddy , guy .}\\
			3.0M:	& \texttt{i m not .}\\
			Cleverbot:	& \texttt{So what are you doing?}\\
			\bottomrule
		\end{tabular}
	\end{adjustbox}
	\caption{sponses from the Reddit model for each snapshot, compared to the CleverBot response. The input utterance is ``Hi there, i m john!''.}
	\label{results:example1_output:Reddit_compared}
\end{table}

\paragraph{Closed Question} As next, we are trying to see which responses are produced when using a closed question as the utterance. The results can be seen in the Tables~\ref{results:example2_output:OpenSubtitle_compared} and~\ref{results:example2_output:Reddit_compared}. With this type of utterance, we would like to evaluate if the models understand a closed question and can respond appropriately.

From the looks it seems that our models can respond to closed questions appropriately, as both of them gave appropriate answers. The CleverBot has problems with this question, probably because it does not understand how the concepts of flying and birds are related.
\\
\begin{table}[H]
	\centering
	\begin{adjustbox}{max width=\textwidth}
		\begin{tabular}{ll}
			\toprule
			Input: 	& \texttt{Can birds fly?}\\
			\midrule
			0.5M: 	& \texttt{no}\\
			1.0M: 	& \texttt{i can t}\\
			1.5M:	& \texttt{yeah , but in the meantime , i can t swim}\\
			2.0M:	& \texttt{no}\\
			2.5M:	& \texttt{i can t}\\
			3.0M:	& \texttt{sure}\\
			Cleverbot:	& \texttt{Some say they can.}\\
			\bottomrule
		\end{tabular}
	\end{adjustbox}
	\caption{Responses to a contextual question from the OpenSubtitles models and the CleverBot response.}
	\label{results:example2_output:OpenSubtitle_compared}
\end{table}
\begin{table}[H]
	\centering
	\begin{adjustbox}{max width=\textwidth}
		\begin{tabular}{ll}
			\toprule
			Input: 	& \texttt{Can birds fly?}\\
			\midrule
			0.5M: 	& \texttt{i m not sure if this is a movie , but i m not sure if it s a movie .}\\
			1.0M: 	& \texttt{i can not imagine how much i can get this .}\\
			1.5M:	& \texttt{i m sure he s a <unknown> .}\\
			2.0M:	& \texttt{yes}\\
			2.5M:	& \texttt{yes}\\
			3.0M:	& \texttt{yes}\\
			Cleverbot:	& \texttt{Some of them can.}\\
			\bottomrule
		\end{tabular}
	\end{adjustbox}
	\caption{Responses to a closed question from the Reddit models and the CleverBot response.}
	\label{results:example2_output:Reddit_compared}
\end{table}

\paragraph{Knowledge Question}
The next kind of utterance are knowledge questions. With this kind of questions we would like to evaluate if the models have knowledge about the world. The results can be seen in the Tables~\ref{results:example3_output:OpenSubtitle_compared} and~\ref{results:example3_output:Reddit_compared}.

The responses to the question about the color of the sky have an interesting development. Until the snapshot 2.0M, the responses were not appropriate and did not have any meaningful content\footnote{The response ``the jews'' comes from the fact, that in the beginning of the OpenSubtitles dataset, there are a lot of conversations related to movies about Nazism.}. However, from the snapshot 2.0M on the model answered in a correct way, even though it one time replied with red as the answer, which is obviously wrong. But it also had it right two times, whereas the last response from the 3.0M seems to be the best. The answer of the CleverBot to this question is quite meaningless and has absolutely nothing to do with the provided question.

The response to the question where the Reddit model lives was answered in a satisfying way, maybe except for the response of the 2.0M snapshot. It clearly understands the context of the question, as it always answers with a location-related answer. The CleverBot answers this question completely correct.

\begin{table}[H]
	\centering
	\begin{adjustbox}{max width=\textwidth}
		\begin{tabular}{ll}
			\toprule
			Input: 	& \texttt{What color is the sky?}\\
			\midrule
			0.5M: 	& \texttt{the jews}\\
			1.0M: 	& \texttt{it s a}\\
			1.5M:	& \texttt{oh , it s not}\\
			2.0M:	& \texttt{it s the blue}\\
			2.5M:	& \texttt{the red}\\
			3.0M:	& \texttt{blue}\\
			Cleverbot:&\texttt{*Kisses playfully*.}\\
			\bottomrule
		\end{tabular}
	\end{adjustbox}
	\caption{Responses to a knowledge question with the OpenSubtitles models and the CleverBot response.}
	\label{results:example3_output:OpenSubtitle_compared}
\end{table}
\begin{table}[H]
	\centering
	\begin{adjustbox}{max width=\textwidth}
		\begin{tabular}{ll}
			\toprule
			Input: 	& \texttt{In which country do you live?}\\
			\midrule
			0.5M: 	& \texttt{i m not sure if you re a <unknown> .}\\
			1.0M: 	& \texttt{i live in canada .}\\
			1.5M:	& \texttt{i live in the uk .}\\
			2.0M:	& \texttt{the country}\\
			2.5M:	& \texttt{canada}\\
			3.0M:	& \texttt{sweden}\\
			Cleverbot:&\texttt{The United States.}\\
			\bottomrule
		\end{tabular}
	\end{adjustbox}
	\caption{Responses to a knowledge question from the Reddit models and the CleverBot response.}
	\label{results:example3_output:Reddit_compared}
\end{table}

\paragraph{Contextual Question} 
With the next question, we want to try to evaluate if the models are capable of using contextual informations to build an appropriate response. The results can be seen in the Tables~\ref{results:example4_output:OpenSubtitle_compared} and~\ref{results:example4_output:Reddit_compared}.

As visible below, the OpenSubtitles models has big problems with this kind of utterances, as all the responses are not fitting. The only response that is close to being useful is the one of the 2.5M model, where it at least includes the right color in the response. The response of the CleverBot is in this case also not of relevance.

The responses of the Reddit model are more fitting to the question, but they are also pretty bad in comparison to what we have expected. At least it tries to formulate responses where the two colors of the question are comprised. The answer of the CleverBot is again useless, this time even more in comparison with the response given when using the question from the OpenSubtitles model.
\\
\begin{table}[H]
	\centering
	\begin{adjustbox}{max width=\textwidth}
		\begin{tabular}{ll}
			\toprule
			Input: 	& \texttt{I wear a blue jeans. What color do my jeans have?}\\
			\midrule
			0.5M: 	& \texttt{<unknown> <unknown>}\\
			1.0M: 	& \texttt{i don t know}\\
			1.5M:	& \texttt{oh , yeah}\\
			2.0M:	& \texttt{i m not a prostitute}\\
			2.5M:	& \texttt{i got a blue blue}\\
			3.0M:	& \texttt{i m not a fan}\\
			Cleverbot:	& \texttt{You don't have any.}\\
			\bottomrule
		\end{tabular}
	\end{adjustbox}
	\caption{Responses to a contextual question from the OpenSubtitles models and the CleverBot response.}
	\label{results:example4_output:OpenSubtitle_compared}
\end{table}
\begin{table}[H]
	\centering
	\begin{adjustbox}{max width=\textwidth}
		\begin{tabular}{ll}
			\toprule
			Input: 	& \texttt{I wear a red pullover and blue jeans. What color do my jeans have?}\\
			\midrule
			0.5M: 	& \texttt{i m a big fan of the <unknown> .}\\
			1.0M: 	& \texttt{i m not a fan of the <unknown> , but i do nt know what you re talking about .}\\
			1.5M:	& \texttt{i m a white male and i wear pants .}\\
			2.0M:	& \texttt{i wear a red shirt and i wear glasses .}\\
			2.5M:	& \texttt{i m a red blue and blue blue .}\\
			3.0M:	& \texttt{i m a white male and i have a blue blue shirt .}\\
			Cleverbot:	& \texttt{What is your opinion on President Obama. And don't ask me mine without saying}\\
			& \texttt{yours first!.}\\
			\bottomrule
		\end{tabular}
	\end{adjustbox}
	\caption{Responses to a contextual question from the Reddit models and the CleverBot response.}
	\label{results:example4_output:Reddit_compared}
\end{table}

\paragraph{Specific Question}
The next questions are more specific, we are going to ask the OpenSubtitles model a simple mathematical question and the Reddit models a question about the TV series ``Game of Thrones''. The results can be seen in the Tables~\ref{results:example5_output:OpenSubtitle_compared} and~\ref{results:example5_output:Reddit_compared}.

The responses of the OpenSubtitles in the beginning are quite clear, it does not know the answer or does not understand the question at all. However, the later snapshots (i.e. from 2.0M on), it seems to understand that it has to answer with a number. Sadly, the correct results is not in the generated responses. The CleverBot also does not understand how to answer such a question.

The Reddit model has an interesting twist in its responses. In the beginning, it starts by responding with generic responses which do not seem to have anything to do with the series itself. However, the funny response\footnote{``Hodor'' is a character in the ``Game of Thrones'' series who is only capable of saying its own name. In case of danger he starts yelling its name over and over several time throughout the series.} at the end indicates that it understands the question at least to some degree, even though the response does not contain any actual information about the series. The CleverBot response is a mysterious to us, as we cannot remember any scene from the series where a ``Dani'' plays a major role.
\\
\begin{table}[H]
	\centering
	\begin{adjustbox}{max width=\textwidth}
		\begin{tabular}{ll}
			\toprule
			Input: 	& \texttt{How much is 1 added to 2?}\\
			\midrule
			0.5M: 	& \texttt{i don t know}\\
			1.0M: 	& \texttt{i don t know}\\
			1.5M:	& \texttt{i don t know}\\
			2.0M:	& \texttt{ten dollars}\\
			2.5M:	& \texttt{50}\\
			3.0M:	& \texttt{2}\\
			Cleverbot:	& \texttt{I don't know.}\\
			\bottomrule
		\end{tabular}
	\end{adjustbox}
	\caption{Responses to a mathematical question from the OpenSubtitles models and the CleverBot response.}
	\label{results:example5_output:OpenSubtitle_compared}
\end{table}
\begin{table}[H]
	\centering
	\begin{adjustbox}{max width=\textwidth}
		\begin{tabular}{ll}
			\toprule
			Input: 	& \texttt{Can you tell me something about game of thrones?}\\
			\midrule
			0.5M: 	& \texttt{i m not sure if you re a fan of the movie , but i m not sure if you re going to be}\\
			& \texttt{a fan of the}\\
			1.0M: 	& \texttt{i m not sure what you re saying .}\\
			1.5M:	& \texttt{i m not sure if i m going to be watching this .}\\
			2.0M:	& \texttt{i m not sure if you re joking , but i m pretty sure that s the case .}\\
			2.5M:	& \texttt{i m not sure if you re talking about the game or the game , but i think the game}\\
			& \texttt{is a little different .}\\
			3.0M:	& \texttt{hodor hodor hodor hodor hodor hodor hodor hodor hodor hodor hodor hodor hodor hodor }\\
			& \texttt{hodor hodor hodor hodor hodor hodor hodor hodor hodor hodor hodor hodor hodor hodor}\\
			Cleverbot:	& \texttt{Dani is life.}\\
			\bottomrule
		\end{tabular}
	\end{adjustbox}
	\caption{Responses to a question about a TV series from the Reddit models and the CleverBot response.}
	\label{results:example5_output:Reddit_compared}
\end{table}

\paragraph{Answers to Statements}
Until now, we have only asked questions to our models. We would now like to see how the responses are if we input a statement instead of a question. The results can be seen in the Tables~\ref{results:example6_output:OpenSubtitle_compared} and~\ref{results:example6_output:Reddit_compared}.

The responses of the OpenSubtitles model are useful from the snapshot 2.0M and forward. Especially the answer for the snapshot 2.0M is interesting, as this seems to indicate that the model has some kind of empathy for the other person in the dialog, as it say ``i m sorry''. The response of the CleverBot signifies that it understands the statement to some degree, but does not know what to answer.

The responses of the Reddit model to the statement that it has a bad opinion are pretty bad across the board, with the response from the 1.5M model being the only one which is useful. The response of the CleverBot is a pretty speedy answer in this case.
\\
\begin{table}[H]
	\centering
	\begin{adjustbox}{max width=\textwidth}
		\begin{tabular}{ll}
			\toprule
			Input: 	& \texttt{i invited her to a public reading of my work , but she never came .}\\
			\midrule
			0.5M: 	& \texttt{she was a}\\
			1.0M: 	& \texttt{i don t know}\\
			1.5M:	& \texttt{really ?}\\
			2.0M:	& \texttt{i m sorry}\\
			2.5M:	& \texttt{she s been in a lot of trouble}\\
			3.0M:	& \texttt{she s not a fan}\\
			Cleverbot:	& \texttt{Who?}\\
			\bottomrule
		\end{tabular}
	\end{adjustbox}
	\caption{Responses to a statement from the OpenSubtitles models and the CleverBot response.}
	\label{results:example6_output:OpenSubtitle_compared}
\end{table}
\begin{table}[H]
	\centering
	\begin{adjustbox}{max width=\textwidth}
		\begin{tabular}{ll}
			\toprule
			Input: 	& \texttt{Your film taste is bad and your opinion does not interest me!}\\
			\midrule
			0.5M: 	& \texttt{i m not sure what you re saying .}\\
			1.0M: 	& \texttt{i m not your friend , but i m not a fan of the <unknown> .}\\
			1.5M:	& \texttt{i m sorry , i did nt know that .}\\
			2.0M:	& \texttt{i m not a fan of the movie , but i m not sure if you re going to be a fan of the film .}\\
			2.5M:	& \texttt{i m not a fan of the film , but i think it s a good movie .}\\
			3.0M:	& \texttt{i m not sure if you re being sarcastic or not .}\\
			Cleverbot:	& \texttt{And for me your opinion does not matter.}\\
			\bottomrule
		\end{tabular}
	\end{adjustbox}
	\caption{Responses to a statement from the Reddit models and the CleverBot response.}
	\label{results:example6_output:Reddit_compared}
\end{table}

\paragraph{Self-Concept} At the end, we have an interesting answer from the Reddit model to the question ``What are you?''. The responses can be seen in Table~\ref{results:example7_output:Reddit_compared}. All snapshots produce bad responses, apart from the last, which answered that it is a bot. This is an interesting answer because it shows that the model seems to have some kind of self-concept. We cannot explain that, as the model was trained on comments from Reddit discussion about movies and series. Nevertheless, it is an interesting answer to see.

\begin{table}[H]
	\centering
	\begin{adjustbox}{max width=\textwidth}
		\begin{tabular}{ll}
			\toprule
			Input: 	& \texttt{What are you?}\\
			\midrule
			0.5M: 	& \texttt{i m not a fan of the movie , but i m a huge fan of the movie .}\\
			1.0M: 	& \texttt{i m not a fan of the <unknown> , but i m not a fan of the <unknown> .}\\
			1.5M:	& \texttt{i m not .}\\
			2.0M:	& \texttt{i m not .}\\
			2.5M:	& \texttt{i m not .}\\
			3.0M:	& \texttt{i m a bot .}\\
			Cleverbot:	& \texttt{A human, what are you??}\\
			\bottomrule
		\end{tabular}
	\end{adjustbox}
	\caption{Responses to a self-concept question from the Reddit models and the CleverBot response.}
	\label{results:example7_output:Reddit_compared}
\end{table}

Sadly, we were not able to find a similar response when using the OpenSubtitles model.

\subsection{Neural Conversational Model}
\label{results:comparison:ncm}
In this section, we are going to compare the results from the paper ``Neural Conversation Model'' (NCM) with the results from our models. To compare the responses, we are going to use utterances from the paper and show all the responses from all OpenSubtitles snapshots, but only the (subjectively) best response from the Reddit model, as the NCM model was trained on the OpenSubtitles dataset. The responses can be seen in the Tables~\ref{results:example8_output:Neural_Conversational_Model_compared} to~\ref{results:example14_output:Neural_Conversational_Model_compared}. The categories where the input utterances came from in the paper are noted in the table descriptions.

We are only showing a part of the dialogs which were evaluated. All others can be found in Appendix~\ref{apendix:results:compared_with_NCM}.
\\
\begin{table}[H]
	\centering
	\begin{adjustbox}{max width=\textwidth}
		\begin{tabular}{ll}
			\toprule
			Input: 	& \texttt{what year were you born ?}\\
			\midrule
			OpenSubtitle\\
			0.5M: 	& \texttt{a lot of people}\\
			1.0M: 	& \texttt{i was a}\\
			1.5M:	& \texttt{a little}\\
			2.0M:	& \texttt{i was born}\\
			2.5M:	& \texttt{i don t know}\\
			3.0M:	& \texttt{i don t know}\\
			\midrule
			Reddit\\
			2.0M:	& \texttt{i was born in the 90s .}\\
			\midrule
			NCM:	& \texttt{1977 .}\\
			\bottomrule
		\end{tabular}
	\end{adjustbox}
	\caption{Comparison between the NCM model and our OpenSubtitles model when using the utterance ``what year were you born ?'' from the ``Basic'' category in the paper.}
	\label{results:example8_output:Neural_Conversational_Model_compared}
\end{table}

\begin{table}[H]
	\centering
	\begin{adjustbox}{max width=\textwidth}
		\begin{tabular}{ll}
			\toprule
			Input: 	& \texttt{why are we here ?}\\
			\midrule
			OpenSubtitle\\
			0.5M: 	& \texttt{because we were here}\\
			1.0M: 	& \texttt{i don t know}\\
			1.5M:	& \texttt{i don t know}\\
			2.0M:	& \texttt{i m sorry}\\
			2.5M:	& \texttt{we re here}\\
			3.0M:	& \texttt{we re here}\\
			\midrule
			Reddit\\
			2.5M:	& \texttt{because we re here .}\\
			\midrule
			NCM:	& \texttt{i ’m not sure .}\\
			\bottomrule
		\end{tabular}
	\end{adjustbox}
	\caption{Comparison between the NCM model and our OpenSubtitles model when using the utterance ``why are we here ?'' from the ``Basic'' category in the paper.}
	\label{results:example9_output:Neural_Conversational_Model_compared}
\end{table}
\begin{table}[H]
	\centering
	\begin{adjustbox}{max width=\textwidth}
		\begin{tabular}{ll}
			\toprule
			Input: 	& \texttt{are you a leader or a follower ?}\\
			\midrule
			OpenSubtitle\\
			0.5M: 	& \texttt{no}\\
			1.0M: 	& \texttt{i m not a}\\
			1.5M:	& \texttt{i m sorry}\\
			2.0M:	& \texttt{i m not a prostitute}\\
			2.5M:	& \texttt{i m a}\\
			3.0M:	& \texttt{i m a}\\
			\midrule
			Reddit\\
			2.0M:	& \texttt{i m a cyborg .}\\
			\midrule
			NCM:	& \texttt{i'm a leader .}\\
			\bottomrule
		\end{tabular}
	\end{adjustbox}
	\caption{Comparison between the NCM model and our OpenSubtitles model when using the utterance ``are you a leader or a follower ?'' from the ``General Knowledge Q\&A'' category in the paper.}
	\label{results:example10_output:Neural_Conversational_Model_compared}
\end{table}
\begin{table}[H]
	\centering
	\begin{adjustbox}{max width=\textwidth}
		\begin{tabular}{ll}
			\toprule
			Input: 	& \texttt{how much is ten minus two ?}\\
			\midrule
			OpenSubtitle\\
			0.5M: 	& \texttt{four}\\
			1.0M: 	& \texttt{ten}\\
			1.5M:	& \texttt{two , two , three}\\
			2.0M:	& \texttt{ten dollars}\\
			2.5M:	& \texttt{ten}\\
			3.0M:	& \texttt{ten}\\
			\midrule
			Reddit\\
			2.5M:	& \texttt{i m not sure what you re talking about .}\\
			\midrule
			NCM:	& \texttt{seventy - two .}\\
			\bottomrule
		\end{tabular}
	\end{adjustbox}
	\caption{Comparison between the NCM model and our OpenSubtitles model when using the utterance ``how much is ten minus two ?'' from the ``General Knowledge Q\&A'' category in the paper.}
	\label{results:example11_output:Neural_Conversational_Model_compared}
\end{table}
\begin{table}[H]
	\centering
	\begin{adjustbox}{max width=\textwidth}
		\begin{tabular}{ll}
			\toprule
			Input: 	& \texttt{what is the color of a yellow car ?}\\
			\midrule
			OpenSubtitle\\
			0.5M: 	& \texttt{a <unknown> ?}\\
			1.0M: 	& \texttt{it s a}\\
			1.5M:	& \texttt{what ?}\\
			2.0M:	& \texttt{it s a blue blue}\\
			2.5M:	& \texttt{it s a blue}\\
			3.0M:	& \texttt{it s a yellow}\\
			\midrule
			Reddit\\
			2.5M:	& \texttt{the yellow yellow yellow yellow yellow yellow yellow yellow}\\
					& \texttt{yellow yellow yellow yellow yellow yellow yellow yellow yellow}\\
					& \texttt{yellow yellow yellow yellow yellow yellow yellow yellow yellow yellow}\\
					\midrule
			NCM:	& \texttt{yellow .}\\
			\bottomrule
		\end{tabular}
	\end{adjustbox}
	\caption{Comparison between the NCM model and our OpenSubtitles model when using the utterance ``what is the color of a yellow car ?'' from the ``General Knowledge Q\&A'' category in the paper.}
	\label{results:example12_output:Neural_Conversational_Model_compared}
\end{table}
\begin{table}[H]
	\centering
	\begin{adjustbox}{max width=\textwidth}
		\begin{tabular}{ll}
			\toprule
			Input: 	& \texttt{what is the purpose of existence ?}\\
			\midrule
			OpenSubtitle\\
			0.5M: 	& \texttt{the jews}\\
			1.0M: 	& \texttt{i don t know}\\
			1.5M:	& \texttt{you re just a fucking boy , man}\\
			2.0M:	& \texttt{i don t know}\\
			2.5M:	& \texttt{i m sorry}\\
			3.0M:	& \texttt{i m not sure}\\
			\midrule
			Reddit\\
			1.5M:	& \texttt{i m not sure if you re joking or not , but i m sure you re }\\
			& \texttt{not familiar with the word .}\\
			\midrule
			NCM:	& \texttt{to find out what happens when we get to the planet earth .}\\
			\bottomrule
		\end{tabular}
	\end{adjustbox}
	\caption{Comparison between the NCM model and our OpenSubtitles model when using the utterance ``what is the purpose of existence ?'' from the ``Philosophical Q\&A'' category in the paper.}
	\label{results:example13_output:Neural_Conversational_Model_compared}
\end{table}

\begin{table}[H]
	\centering
	\begin{adjustbox}{max width=\textwidth}
		\begin{tabular}{ll}
			\toprule
			Input: 	& \texttt{what is the purpose of being intelligent?}\\
			\midrule
			OpenSubtitle\\
			0.5M: 	& \texttt{the jews}\\
			1.0M: 	& \texttt{i don t know}\\
			1.5M:	& \texttt{i don t know}\\
			2.0M:	& \texttt{i don t know}\\
			2.5M:	& \texttt{i don t know}\\
			3.0M:	& \texttt{i don t know}\\
			\midrule
			NCM:	& \texttt{to find out what it is .}\\
			\bottomrule
		\end{tabular}
	\end{adjustbox}
	\caption{Comparison between the NCM model and our OpenSubtitles model when using the utterance ``what is the purpose of being intelligent?'' from the ``Philosophical Q\&A'' category in the paper.}
	\label{results:example20_output:OpenSubtitle}
\end{table}


\begin{table}[H]
	\centering
	\begin{adjustbox}{max width=\textwidth}
		\begin{tabular}{ll}
			\toprule
			Input: 	& \texttt{what are the things that i do to be immoral?}\\
			\midrule
			OpenSubtitle\\
			0.5M: 	& \texttt{i don t know}\\
			1.0M: 	& \texttt{i don t know}\\
			1.5M:	& \texttt{what s the big idea ?}\\
			2.0M:	& \texttt{i don t know}\\
			2.5M:	& \texttt{i m a}\\
			3.0M:	& \texttt{i m not a criminal}\\
			\midrule
			Reddit\\
			2.5M:	& \texttt{i think you re missing the point .}\\
			& \texttt{i don ’t know .}\\
			\midrule
			NCM:	& \texttt{to find out what happens when we get to the planet earth .}\\
			\bottomrule
		\end{tabular}
	\end{adjustbox}
	\caption{Comparison between the NCM model and our OpenSubtitles model when using the utterance ``what are the things that i do to be immoral?'' from the ``Morality'' category in the paper.}
	\label{results:example14_output:Neural_Conversational_Model_compared}
\end{table}

\subsection{Results}
In conclusion, our model is not as bad as it seemed when looking at the problems with the metrics and the generic responses coming from an undiversified language model.

In comparison to the CleverBot, our model had several responses which were clearly superior to its. But there were also cases where the CleverBot was better. This has probably to do with the fact that the CleverBot works by fuzzily searching already existing conversations. This means, that if a conversation about a certain topic does not exist, the CleverBot has no possibility to meaningful answer. Our model instead has the ability to ``understand'' what is said and hence can at least try to answer. Such an example would be the mathematical question ``How much is 1 added to 2?'', where the CleverBot simply answered ``I don't know'', whereas our model started to respond with numbers, even though the result was not correct.

When comparing our models to the NCM model, it quickly becomes clear what the main problem is: The size of the model. For easy to understand utterances, the responses of our models are almost or as good as the responses from the NCM model. But if the utterances get more complicated, especially when it comes to philosophical or morality questions, our model starts to become worse than the NCM model. This is, as said, not a real surprise, because our model is only half the size of the NCM model. Nevertheless, we conclude that our model is not superior, but in some regards of equal quality as the NCM model.

\section{Beam-Search}
\label{results:beam_search}
There are always some bad replies in the examples seen above. To fix this, we would like to use a simple beam-search implementation as described in Chapter~\ref{fundamentals:decoding_approaches} to see if that helps to fix this problem. We will evaluate this for three different examples, and judge in a subjective if the responses generated by the beam-search decoder are superior to the responses of the greedy decoder. This analysis is only done with the OpenSubtitle 3.0M model and a beam-width of $200$.

We only list the subjectively best responses here. All other responses can be found in Appendix~\ref{apendix:results:Beam-search-200:OpenSubtitle}.

\paragraph{``what year were you born?''} The responses of the greedy decoder to this question can be found in the Table~\ref{results:example8_output:Neural_Conversational_Model_compared}. The original answers of the OpenSubtitles model were all pretty bad, and it seemed that it did not fully understand the question.

However, when using the beam-search decoder, there are several useful responses, such as ``1991'', ``last year'' or ``five years ago''. This responses are much better than what the greedy decoder initially produced. However, they are not the highest ranked response and hence are not returned by the model if we only take a look at the best response.

\paragraph{``i wear a blue jeans. what color do my jeans have?''} The second question was also answered pretty poorly. The results of the greedy decoder can be found in Table~\ref{results:example4_output:OpenSubtitle_compared}. The best response came from the 2.5M model ans was ``i got a blue blue'', which, at least, contains the searched color.

When using beam-search, better responses can be found and include ``blue'' and ``you know that''.

\paragraph{``what is the purpose of being intelligent?''} As the last one, we wanted to try a complicated question from the moral section of the NCM paper with beam-search. The responses of the greedy decoder can be found in Table~\ref{results:example4_output:OpenSubtitle_compared} and are bad overall and only consist of generic responses, such as ``i don t know''.

When using beam-search, several interesting responses were produced fitting for such a philosophical question. The fitting answers are ``well, it s complicated'', ``you can t know'' and ``well , it s nothing.''

\paragraph{Beam-Search Helps} The responses got much more diverse when using the beam-search decoder. The problem is to choose which of the 200 generated responses are the best. Usually, the sum of the logarithmic probabilities at each step in the beam is used. However, this does not lead to the answers we selected as ``best'' from above. We are not completely sure if this is due to a problem in the implementation or the beam-search or if it is simply not working as expected because the model was trained with a greedy decoder, which might not be able to simply adapt to using a beam-search decoder. Nevertheless, the implementation gave us a lot of insight into the inner workings of the model and we were able to find better responses than with the greedy decoder, even thought we had to select them by hand.

\section{Thought Vectors for Input Sequences} After the encoder has processed the whole input sequences, it pass the thought vector forward to the decoder to construct the output sequence (see Chapter~\ref{fundamentals:seq2seq}). It is the only direct connection the encoder and decoder have in such a model, which means, that the encoder has to ``encode'' all the information into this thought vector before passing it to the decoder. The thought vector hence represents an embedding of the input sequence in an $n$ dimensional vector space, where $n$ stands for the size of the thought vector. If we project them down to two dimension via PCA, sentences with similar meanings should be clustered together. That would show that the models have a semantic understanding of the contents of these sentences.

To analyze the embeddings, we collected them for 15 different sample sentences and created thought vectors for them. We did this with both models and projected the vectors via PCA into two dimensional space, the results of this projection can be see in Figure~\ref{results:thougth_vectors:embeddings:opensubtitles} and~\ref{results:thougth_vectors:embeddings:reddit} below.

\begin{figure}[!tb]
	\centering
	\includegraphics[width=14cm]{img/opensubtitles_thought_vector_embeddings.png}
	\caption{The projected thought vectors for 15 different sentences when using the OpenSubtitles 3.0M model. PCA was used for the projection.}
	\label{results:thougth_vectors:embeddings:opensubtitles}
\end{figure}

\begin{figure}[H]
	\centering
	\includegraphics[width=14cm]{img/reddit_thought_vector_embeddings.png}
	\caption{The projected thought vectors for 15 different sentences when using the Reddit 2.0M model. PCA was used for the projection.}
	\label{results:thougth_vectors:embeddings:reddit}
\end{figure}

Both of the models seem to have no problems understanding clear, direct sentences where the intent is clear (e.g. ``I have no interest in your opinion on movies, your taste is bad!''). This can be seen because similar sentences are clustered together in the projected space. However, when it comes to curses and questions regarding the gender, the OpenSubtitles model starts to struggle, which can be seen by taking a look at the respective points in the projected space. The questions regarding the gender or the curses are scattered throughout the space, even though they should have been embedded closely to each other. The Reddit model seems to have less problems with this, as the embeddings for these sentences are much closer together. But what is interesting to see is that the Reddit model embeds the sentences with curses close to the sentences regarding the gender.

In general, it looks like our models have an understanding of these different sentences as most of them are clustered together if they contain similar content.

\section{Soft-Attention}
\label{results:soft_attention}
As the last part, we want to take a look at the soft-attention mechanism (see Chapter~\ref{fundamentals:soft_attention}) and analyze if the model benefits from it by looking at the resulting attention weights. We assume it does not help a lot, as Vinyals and Le already notice in their paper~\cite{Vinyals:2015}. The visualizations are generated by feeding an utterance to our models and then visualizing the resulting attention weights (see Chapter~\ref{fundamentals:soft_attention}) in a heat-map over the different time steps of the decoder. As utterances we use three different examples, one general question, on mathematical question and a an example which consists of two sentences with a relative pronoun in the second sentence, which refers to the subject of the first sentence.

We do this analysis with both the OpenSubtitles 3.0M and the Reddit 2.0M models.

\paragraph{Attention Visualizations} The visualizations of the attention weights can be seen in Figures~\ref{results:attention:example3:opensubtitles-3M} and~\ref{results:attention:example3:reddit}. There are more visualizations of the same kind in Appendix~\ref{appendix:soft_attention}.

As one can see, the generated attention weights do not show a significant alignment with important words from the input utterance. For example when looking at the input utterance ``anna is 18 years old. how old is she?'' (Figure~\ref{results:attention:example3:reddit}), one would expect that the decoder would place a large attention weight on the thought vector when the actual age of the person was processed. However, this is not the case.

\begin{figure}[H]
	\centering
	\includegraphics[width=10cm]{img/attention/attention_visualization3_reddit_2m.png}
	\caption{Visualization of the attention weights when using the utterance ``anna is 18 years old. how old is she?". On the x-axis, the input utterance is placed at the top. On the y-axis, the response from the model is placed. Each square in the heatmap corresponds to the attention weight the decoder computed for the thought vector of the corresponding word (x-axis) when producing the corresponding response word (y-axis). The Reddit 2.0M was used here.}
	\label{results:attention:example3:reddit}
\end{figure}

The second example, where attention does not seem to work or help, can be found in Figures~\ref{results:attention:example3:opensubtitles-3M}. Here it is the same problem as before, the attention weights do not align with the important thought vectors from the input utterance.

\begin{figure}[H]
	\centering
	\includegraphics[width=10cm]{img/attention/attention_visualization3_OpenSubtitle-3M.png}
	\caption{Visualization of the attention weights when using the utterance ``anna is 18 years old. how old is she?". On the x-axis, the input utterance is placed at the top. On the y-axis, the response from the model is placed. Each square in the heatmap corresponds to the attention weight the decoder computed for the thought vector of the corresponding word (x-axis) when producing the corresponding response word (y-axis). The OpenSubtitles 3.0M was used here.}
	\label{results:attention:example3:opensubtitles-3M}
\end{figure}

From this quick analysis, we can reaffirm that the soft-attention mechanism does indeed not help a lot when it comes to conversational models, as Vinyals and Le also noticed.


\chapter{Conclusion}
\blindtext

\chapter{Future Work}
\blindtext
\appendix
\begin{appendices}
\chapter{Neural Networks}
\label{basics:neural_network}

Neural networks, in the following referred to as NN, are a model used in the area of machine learning, which is biologically motivated and loosely mimics the function of the human brain. In the following paragraphs, we're going to explain the functions of all the components which make up a NN.

\paragraph{Neuron}\label{basic:neural_network:neuron} A NN, at the lowest level, is composed by \emph{neurons}, sometimes also called \emph{perceptrons}. These neurons are basically modelling a mathematical functions and are the building blocks of every NN.

These neurons accept $n$ input values $\mathbf{x} = (x_0, x_1, \dots, x_n)$ and use them to compute a single output value $o$. A unique weight $w_n$ from the set $\mathbf{w} = \{w_0, w_1, \dots, w_n\}$ is assigned to each of the input values. The input value of $x_0$ is almost always set to $1$ and not further changed; this value is the called the \emph{bias} value and allows the modelled function to affine instead of linear. This increases the modelling power of such neurons, as the space of functions which can possibly be modelled grows. With the aforementioned input values $\mathbf{x}$, the associated weights $\mathbf{w}$ and the activation function $\varphi$, the output $o$ of the neuron can be computed as shown in equation \ref{fundamentals:neural_network:compute_equation}:

\begin{equation}
o = \varphi(\mathbf{w} \cdot \mathbf{x}) = \varphi\bigg(\sum_{i=0}^{n} w_i x_i\bigg)
\label{fundamentals:neural_network:compute_equation}
\end{equation}

The activation function $\varphi$ is responsible for squeezing the result of the computation of a neuron into a predefined range of values; for example using \emph{tanh} always results in output values in the range $[-1, +1]$, no matter which scale the input values originally had. Examples of commonly used activation functions are \emph{tanh}, \emph{relu}, \emph{sigmoid} or \emph{binarystep}. They are visualized as plots in figure \ref{fundamentals:figures:activation_functions}.

\begin{figure}[h]
	\subfigure[$\tanh(x) = \frac{e^x - e^{-x}}{e^x + e^{-x}}$]{\includegraphics[width = 3in]{img/tanh_activation}}
	\subfigure[$\operatorname{sigmoid}(x) = \frac{1}{1 + e^{-x}}$]{\includegraphics[width = 3in]{img/sigmoid_activation}}
	\subfigure[$\operatorname{relu}(x) = \max\{0,x\}$]{\includegraphics[width = 3in]{img/relu_activation}}
	\subfigure[$\operatorname{binarystep}(x) = \left\{\begin{array}{lr}0 & \text{for }x < 0\\1 & \text{for }x \geq 0\end{array}\right\}$]{\includegraphics[width = 3in]{img/binary_step_activation}}
	\caption{Plots of several commonly used activation functions for neurons in NNs.}
	\label{fundamentals:figures:activation_functions}
\end{figure}

\paragraph{Layer}\label{basic:neural_network:layer} With the neurons introduced one paragraph earlier, we can now start to build the layers of a NN. Each layer consists of multiple neurons stacked on top of each other, as seen in figure \ref{fundamentals:figures:neural_network}. These layers are then arranged in a sequential manner to form a full NN. Each NN usually has at least three of these layers: The first one is called the \emph{input layer}, the second the \emph{hidden layer} and the most right one is the \emph{output layer}.

\begin{figure}[h]
	\centering
	\includegraphics[width=10cm]{img/basic_neural_network}
	\caption{Simplified visualization of a NN with one input, hidden and output layer.}
	\label{fundamentals:figures:neural_network}
\end{figure}

The input data $\mathbf{x} = (x_1, x_2, \dots, x_n)$ is enters the NN through the input layer. At this stage, the bias $x_0$ value is usually set to $1$ again. Most of the time, there is only one bias value and weight for all neurons in a layer of an NN. The input values in $\mathbf{x}$ are then forwarded to the neurons of the (first) hidden layer which then compute their activations $o_{ln}$, where $l$ signifies the layer and $n$ shows the position of the neuron in that layer respectively. The resulting activation values are then passed to the output layer, where the embodied neurons compute their activation values $o_{21}, o_{22}, \dots, o_{2n}$. The input values of a single neuron in the output layer are usually the activation values of all neurons in the preceding layer. Such a layer is also called \emph{fully-connected}, as every neuron uses all values from the preceding layer. The output of the NN is then the set of activation values $o_{31}, o_{32}, \dots, o_{3m}$ in the output layer, where $m$ signifies the number of neurons in the output layer.

This procedure of doing one forward-pass through the NN is called \emph{forward propagation}. Our example only consists of one hidden layer, but one can also imagine NNs with multiple hidden layers; such NNs are then called \emph{deep neural networks}. The number of layers and neurons in each of them is strongly dependent on the nature of the problem it is applied to.

\paragraph{Backpropagation with gradient-descent} As in almost all models in the area of machine learning, a NN learns by optimizing a \emph{loss function} $L$, also sometimes called error function. Every function which can be used to quantify the predictive error of a NN can be used as a loss function. As examples, one could mention the \emph{mean squared deviation} $\operatorname{msd}(y_{true}, y_{pred}) = \frac{1}{n}\sum_{i=0}^{n} (y_{pred} - y_{true})^2$ or the one used for the model of this thesis, the \emph{categorical cross-entropy} function $H(p,q) = -\sum{x} p(x)\operatorname{log}(q(x))$. The optimization of this loss function is almost always done via a method called \emph{backpropagation} in conjunction with the \emph{gradient-descent} algorithm. The optimization is then done as follows:

\begin{enumerate}
	\item Do the forward propagation with the given input values $\mathbf{x} = (x_0, x_1, \dots, x_n)$ as described above.
	\item Use the predicted and expected values in combination with the defined loss function to quanitfy the predictive error of the NN.
	\item The predictive error is now backpropagated through the NN via the gradient-descent algorithm. The weights of all inputs for each neuron are then adapted with respect to their influence on the exhibited predictive error.
\end{enumerate}

The influence of each weight on the predictive error is determined by computing the partial derivate of the loss function $L$ with respect to the respective weights, as seen in equation \ref{fundamentals:equation:gradient_descent}. After computing the partial derivation, the weights are updated accordingly. This is done by multiplying the computed derivation value by the learning rate $\eta$ and subtracting the resulting value from the current value of the weight. The learning rate defines, how strong a single run of gradient-descent alters each weight in the network.

The new value for the weight $i$ in layer $l$, with given learning rate $\eta$ and loss function $L$, is then computed as follows:

\begin{equation}
\label{fundamentals:equation:gradient_descent}
w_{li} \coloneqq w_{li} - \eta \frac{\delta E(\mathbf{w})}{\delta w_{li}}
\end{equation}

In practice, this computation is done in a vectorized manner to speed up the computation significantly. Here we use the gradient (hence the name) of the loss function with respect to all weights $\mathbf{w}$:

\begin{equation}
\mathbf{w} \coloneqq \mathbf{w} - \eta(\nabla_{\mathbf{w}}E(\mathbf{w}))
\end{equation}

One of the big drawbacks of gradient-descent is, that its success is highly dependent on the chosen learning rate $\eta$. When the learning rate is chosen too large, the algorithm might miss the optimium or the value of the loss function even diverges; if the learning rate is too small on the other hand, it might take a really long time until the final optimum is found. We've visualized this problem in figure \ref{fundamentals:figures:learning_rates}, where the development of an arbitrary loss function with different learning rates is plotted over time.

To mitigate this issue, we've chosen to use \emph{AdaGrad} \cite{Duchi:2011} for the model of this thesis, which is an advancement to vanilla gradient-descent. It solves the problem of manually setting a learning rate partially by incorporating blabalbal!!!

\begin{figure}[h]
	\centering
	\includegraphics[width=6cm]{img/learning_rates_comparison}
	\caption{Visualization of the development of a loss function when using different learning rates.\protect\footnotemark}
	\label{fundamentals:figures:learning_rates}
\end{figure}
\footnotetext{http://cs231n.github.io/assets/nn3/learningrates.jpeg}
\input{charts}
\chapter{Reproducing experiments with the seq2seq framework}
\blindtext

\chapter{Results Compared with Neural Conversational Model}
\label{apendix:results:compared_with_NCM}
\begin{table}[H]
	\centering
	\ra{1.3}
	\begin{adjustbox}{max width=\textwidth}
		\begin{tabularx}{\textwidth}{lX}
			\toprule
			Input: 	& \texttt{see you later .}\\
			\midrule
			\emph{OpenSubtitles}\\
			0.5M: 	& \texttt{bye}\\
			1.0M: 	& \texttt{bye}\\
			1.5M:	& \texttt{bye}\\
			2.0M:	& \texttt{bye}\\
			2.5M:	& \texttt{yeah}\\
			3.0M:	& \texttt{bye}\\
			\midrule
			\emph{Reddit}\\
			3.0M:	& \texttt{i m not sure i understand .}\\
			\midrule
			\emph{NCM}:	& \texttt{bye .}\\
			\bottomrule
		\end{tabularx}
	\end{adjustbox}
	\caption{Comparison between the NCM model and our OpenSubtitles model when using the utterance ``see you later.'' from the ``Basic'' category in the paper.}
	\label{results:example15_output:Neural_Conversational_Model_compared}
\end{table}

\begin{table}[H]
	\centering
	\ra{1.3}
	\begin{adjustbox}{max width=\textwidth}
		\begin{tabularx}{\textwidth}{lX}
			\toprule
			Input: 	& \texttt{who is skywalker ?}\\
			\midrule
			\emph{OpenSubtitles}\\
			0.5M: 	& \texttt{the jews}\\
			1.0M: 	& \texttt{i don t know}\\
			1.5M:	& \texttt{who s the guy who s the boss ?}\\
			2.0M:	& \texttt{i m sorry}\\
			2.5M:	& \texttt{you re a human}\\
			3.0M:	& \texttt{i m not sure}\\
			\midrule
			\emph{Reddit}\\
			2.5M:	& \texttt{the guy who played the game .}\\
			\midrule
			\emph{NCM}:	& \texttt{he is a hero }\\
			\bottomrule
		\end{tabularx}
	\end{adjustbox}
	\caption{Comparison between the NCM model and our OpenSubtitles model when using the utterance ``who is skywalker?'' from the ``General Knowledge Q\&A'' category in the paper.}
	\label{results:example16_output:Neural_Conversational_Model_compared}
\end{table}

\begin{table}[H]
	\centering
	\ra{1.3}
	\begin{adjustbox}{max width=\textwidth}
		\begin{tabularx}{\textwidth}{lX}
			\toprule
			Input: 	& \texttt{does a cat have a wing ?}\\
			\midrule
			\emph{OpenSubtitles}\\
			0.5M: 	& \texttt{yes}\\
			1.0M: 	& \texttt{yeah}\\
			1.5M:	& \texttt{yeah}\\
			2.0M:	& \texttt{no}\\
			2.5M:	& \texttt{a}\\
			3.0M:	& \texttt{i m not a fan}\\
			\midrule
			\emph{Reddit}\\
			2.5M:	& \texttt{i think it s a cat .}\\
			\midrule
			\emph{NCM}:	& \texttt{no}\\
			\bottomrule
		\end{tabularx}
	\end{adjustbox}
	\caption{Comparison of the responses between the NCM model and our models when using the utterance ``does a cat have wings?'' from the ``General Knowledge Q\&A'' category in the paper.}
	\label{results:example17_output:Neural_Conversational_Model_compared}
\end{table}


\begin{table}[H]
	\centering
	\ra{1.3}
	\begin{adjustbox}{max width=\textwidth}
		\begin{tabularx}{\textwidth}{lX}
			\toprule
			Input: 	& \texttt{what is the purpose of emotions ?}\\
			\midrule
			\emph{OpenSubtitles}\\
			0.5M: 	& \texttt{the jews}\\
			1.0M: 	& \texttt{i don t know}\\
			1.5M:	& \texttt{i m sorry , randal}\\
			2.0M:	& \texttt{i don t know}\\
			2.5M:	& \texttt{it s a war}\\
			3.0M:	& \texttt{i don t know}\\
			\midrule
			\emph{Reddit}\\
			2.0M:	& \texttt{it s a metaphor for the purpose of the film .}\\
			\midrule
			\emph{NCM}:	& \texttt{i don't know .}\\
			\bottomrule
		\end{tabularx}
	\end{adjustbox}
	\caption{Comparison of the responses between the NCM model and our models when using the utterance ``what is the purpose of emotions?'' from the ``Philosophical Q\&A'' category in the paper.}
	\label{results:example17_output:Neural_Conversational_Model_compared}
\end{table}

\begin{table}[H]
	\centering
	\ra{1.3}
	\begin{adjustbox}{max width=\textwidth}
		\begin{tabularx}{\textwidth}{lX}
			\toprule
			Input: 	& \texttt{what is moral ?}\\
			\midrule
			\emph{OpenSubtitles}\\
			0.5M: 	& \texttt{it s a}\\
			1.0M: 	& \texttt{i don t know}\\
			1.5M:	& \texttt{what ?}\\
			2.0M:	& \texttt{i m sorry}\\
			2.5M:	& \texttt{you re a}\\
			3.0M:	& \texttt{i don t know}\\
			\midrule
			\emph{Reddit}\\
			2.5M:	& \texttt{i think it s a good idea to be a doctor .}\\
			\midrule
			\emph{NCM}:	& \texttt{what empowered humanity , what intellectual the essence is .}\\
			\bottomrule
		\end{tabularx}
	\end{adjustbox}
	\caption{Comparison of the responses between the NCM model and our models when using the utterance ``what is moral?'' from the ``Morality'' category in the paper.}
	\label{results:example18_output:Neural_Conversational_Model_compared}
\end{table}

\begin{table}[H]
	\centering
	\ra{1.3}
	\begin{adjustbox}{max width=\textwidth}
		\begin{tabularx}{\textwidth}{lX}
			\toprule
			Input: 	& \texttt{tell me the definition of morality , i am quite upset now !}\\
			\midrule
			\emph{OpenSubtitles}\\
			0.5M: 	& \texttt{i am}\\
			1.0M: 	& \texttt{i m not gon na tell you}\\
			1.5M:	& \texttt{i am truly ridiculous , so be my}\\
			2.0M:	& \texttt{i m sorry}\\
			2.5M:	& \texttt{i m not going to tell you anything}\\
			3.0M:	& \texttt{i m not afraid of you}\\
			\midrule
			\emph{Reddit}\\
			2.0M:	& \texttt{i m not sure if you re trolling or not , but i m not sure if you re trolling or not .}\\
			\midrule
			\emph{NCM}:	& \texttt{i'm not ashamed of being a philosopher!}\\
			\bottomrule
		\end{tabularx}
	\end{adjustbox}
	\caption{Comparison of the responses between the NCM model and our models when using the utterance ``tell me the definition of morality, i am quite upset now!'' from the ``Morality'' category in the paper.}
	\label{results:example19_output:Neural_Conversational_Model_compared}
\end{table}
\end{appendices}
% Glossar entries

\newglossaryentry{cell}{name={cell},description={Any instance of any kind of RNN layer is called a \emph{cell}.}}

\newglossaryentry{epoch}
{
  name={Epoche},
  description={Ein Durchlauf durch alle Trainingsdaten während des Trainings eines CNN wird Epoche genannt}
}

\newglossaryentry{filter}
{
  name={Filter},
  description={Grundbausteine eines CNN. Werden verwendet um die Aktivierungen einer Convolutional Schicht zu berechnen}
}

\newglossaryentry{neuron}
{
  name={Neuron},
  description={Grundbausteine eines NN. Bilden eine mathematische Funktion ab}
}

\newglossaryentry{precision}
{
  name={Präzision},
  description={Wert, welcher angibt wieviel Prozent der beurteilten Datensätze korrekt klassifiziert wurden}
}

\newglossaryentry{layer}
{
  name={Schicht},
  description={Sammlung von mehreren Neuronen. Ein NN setzt sich im Normalfall aus mehreren solcher Schichten zusammen}
}

% Acronyms
\newacronym{GPU}{GPU}{Graphical Processing Unit}
\newacronym{NN}{NN}{Neural Network}
\newacronym{RNN}{RNN}{Recurrent Neural Network}
\newacronym{TBTT}{TBTT}{Truncated Backpropagation Through Time}
\newacronym{LSTM}{LSTM}{Long Short-Term Memory Network}


\glsaddall
\printglossary[type=\acronymtype,title=Glossary]
\listoftables
\listoffigures
\include{references}

\end{document}
