\chapter{Software System}
We had to develop a software system which allows us to conduct experiments for this thesis. The following paragraphs will therefore give a brief explanation on how we've decided to develop the system in order to enable us to conduct experiments with seq2seq models.

\begin{itemize}
	\item Introduce basic idea behind system
	\item Tell that we're using TensorFlow
\end{itemize}
\section{Requirements}
\begin{itemize}
	\item Parameterized system
	\item Easy evaluation of trained models
	\item Allow us to analyze the results
\end{itemize}
\section{Development of the System}
\begin{itemize}
	\item Why we've switch from Keras to Tensorflow
	\item Beginning was really hard, as TF is much more low level than keras
	\item Had huge problems due to different seq2seq APIs with different levels of documentation, bug-freenes and future-support.
	\item Google Seq2Seq Tool -> Inference broken
	\item Settled with legacy API as we know this one is working
\end{itemize}
\section{Model Validation Checks}
\begin{itemize}
	\item Link to previous paragraph and explain why we had to introduce model validation checks.
	\item Explain the overfitting test
	\item Explain the copy test
\end{itemize}

\section{Scripts}
\begin{itemize}
	\item Explain what we've used scripts for
\end{itemize}
\section{Web-UI}
\begin{itemize}
	\item Small description of "talk to model" GUI with screenshots
\end{itemize}
\section{Operating System \& Software Packages}
\begin{itemize}
	\item Declare all dependencies and how the system has to be setup in order to use the software
\end{itemize}
\section{Hardware}
\begin{itemize}
	\item Explain the hardware on the GPU-Cluster, as all experiments were run there
\end{itemize}

\iffalse
\section{Technischer Aufbau}
\label{technical_setup}
Im folgenden Abschnitt wird der technische Aufbau erläutert, welcher verwendet wird, um die in Kapitel \ref{sec:Experimente_Resultate} beschriebenen Experimente durchzuführen. Eine Beschreibung zur Verwendung des Systems befindet sich in Anhang \ref{appendix:software_usage}.

\paragraph{Vorarbeiten}
\label{technichal_setup:prework}
Der Grundaufbau der verwendeten Software wurde vom InIT mithilfe von \texttt{keras}\footnote{https://keras.io/} implementiert und zur Durchführung dieser Arbeit zur Verfügung gestellt. Im Rahmen dieses Grundaufbaus wurden die folgenden Funktionalitäten bereits implementiert:

\begin{itemize}[noitemsep]
	\item Implementation des CNN in \texttt{keras} und verwendung von \texttt{theano} \cite{theanoCitShort} als Backend für die \gls{GPU}s.
	\item Implementation von Evaluations-Metriken.
	\item Skripte mit den folgenden Funktionalitäten: Trainieren des CNN, Laden von TSV Dateien, Vorverarbeiten von Word-Embeddings.
\end{itemize}

\paragraph{Anforderungen}
\label{technical_setup:requirements}
Ein zu implementierendes System, mit welchem die Experimente durchgeführt werden können, soll die folgenden Eigenschaften aufweisen:

\begin{itemize}
	\item \textbf{Parametrisierbarkeit}: Dadurch, dass eine grosse Anzahl kleiner Experimente durchgeführt werden muss, soll das System die Möglichkeit bieten, Experimente parametrisiert durchzuführen.
	\item \textbf{Wiederholbarkeit}: Experimente sollen mit einem minimalen Mehraufwand mehrfach durchgeführt werden können.
	\item \textbf{Übersichtlichkeit}: Resultate der Experimente sollen übersichtlich und einfach zugänglich sein.
	\item \textbf{Auswertbarkeit}: Resultate sollen automatisiert ausgewertet werden können.
\end{itemize}

\paragraph{Funktionalität}
\label{technical_setup:functionality}
Um ein System, welches die oben beschriebenen Anforderungen erfüllt zu erhalten, werden die folgenden Komponenten implementiert:

\begin{itemize}
	\item \textbf{Executor}: Der \emph{Executor} ist zuständig für das Training der CNNs mithilfe von \texttt{keras}. Beim Start akzeptiert er die Konfiguration als Parameter. Das Experiment wird mit dem Laden der benötigten Daten und dem anschliessenden Training des CNN gestartet. Am Ende jeder Epoche wird das aktuelle CNN auf den Validierungsdaten getestet und die konfigurierten Metriken ausgewertet. Diese werden am Ende zusammen mit dem trainierten CNN (Gewichte im HDF5-Format\footnote{https://support.hdfgroup.org/HDF5/}, das CNN Model als JSON) in einen für das Experiment vorgesehenen Ordner gespeichert. Die Metriken werden ebenfalls in dem dafür vorgesehenen Ordner abgespeichert.
	\item \textbf{Config Management}: Experimente werden über Konfigurationen im JSON-Format\footnote{http://www.json.org/} parametrisiert. Über diese Konfiguration können viele wichtige Parameter für die Ausführung festgelegt werden, so zum Beispiel: Anzahl Epochen, Trainings- und Validierungsdaten, Parameter für die k-fold Cross-Validation oder auch bereits trainierte Modelle können geladen werden. Detailierte Erläuterungen zu den einzelnen Parametern können im Anhang \ref{appendix:software_usage} gefunden werden.
	\item \textbf{DataLoader}: Mithilfe des \emph{DataLoader} können Trainings- und Validierungsdaten im TSV\footnote{https://reference.wolfram.com/language/ref/format/TSV.html} Dateiformat geladen werden. Die zu ladenden Daten können dabei aus einer oder mehreren TSV-Dateien stammen. Im Falle, dass mehrere TSV Dateien angegeben werden, kann über die Konfiguration das Verhältnis angegeben werden, in welchem die Daten aus den einzelnen Dateien verschmischt werden sollen.
	\item \textbf{Skripte}: Die Auswertung der einzelnen Experimente geschieht über dafür erstellte Skripte.
	\item \textbf{Weboberfläche}: Auf die Resultate der Experimente kann über eine eigens dafür entwickelte Weboberfläche zugegriffen werden. Ausserdem besteht die Möglichkeit Plots über die Metriken, welche während des Trainings- und Validierungsprozess gesammelt werden, zu erstellen.
	
\end{itemize}
Die oben beschriebenen Komponenten erlauben es, Experimente mittels JSON Konfigurationen zu starten und den gesamten Trainings- und Validierungsprozess mittels Metriken zu überwachen und zu dokumentieren.

\paragraph{Skripte}
\label{technical_setup:scripts}
Für die Durchführung der Experimente wurden diverse Skripte erstellt, um die Handhabung zu vereinfachen und Auswertungen zu ermöglichen. Die Liste der implementierten Scripts umfasst unter anderem die folgenden:

\begin{itemize}[noitemsep]
	\item Erstellen von Plots der Lernkurven und Metriken
	\item Erstellen von Word-Embeddings über einen Text-Corpus
	\item Erstellen von Statistiken zu Trainings- und Validierungsdaten
	\item Vorverarbeitung von Trainingsdaten für die Distant-Phase
	\item Erstellen von Visualisierungen von Word-Embeddings mittels PCA
	\item Diverse Wartungsskripte zur Generierung und Verwaltung von Experimenten
\end{itemize}

\paragraph{Weboberfläche}
\label{technical_setup:webgui}
Um die dritte Anforderung nach Übersichtlichkeit und Auswertbarkeit zu erfüllen, wird eine Weboberfläche umgesetzt, mit welchem die Parameter und Resultate aller durchgeführten Experimente übersichtlich und an einem Ort zur Verfügung gestellt werden. Für die Implementation wird die \texttt{python}\footnote{https://www.python.org/} Bibliothek \texttt{flask}\footnote{http://flask.pocoo.org/} verwendet.

Zur Auswertung der Experimente stehen drei Funktionen zur Verfügung:
\begin{itemize}
	\item Die Oberfläche gewährt Zugriff auf alle JSON Konfigurationen, welche zu einem Experiment gehören. Dazu zählen die Konfiguration selbst, die gespeicherten Trainings- und Validierungsmetriken und das \texttt{keras} Model des CNN.
	\item Mittels der Plotting Funktion können Plots von Trainings- und Validierungsmetriken erstellt werden.
	\item Die gespeicherten Validierungs- und Trainingsmetriken können mithilfe von \texttt{math.js}\footnote{http://mathjs.org/} direkt im Browser ausgewertet werden.
\end{itemize}

\paragraph{Betriebssystem \& Softwarepakete}
\label{technical_setup:software}
Alle Experimente werden mit dem oben beschriebenen Software-System durchgeführt. Auf den beiden verwendeten Computer-Systemen wird als Betriebssystem Ubuntu 16.04 installiert. Dazu werden \texttt{python} in der Version 3.5.2, Nvidia GPU Treiber und \texttt{cuda}\footnote{https://developer.nvidia.com/cuda-toolkit} in der Version 8.0 als Abhängigkeiten von \texttt{theano} und \texttt{keras} installiert.

\paragraph{Hardware}
\label{technichal_setup:hardware}
Zur Durchführung der Experimente werden zwei unterschiedliche Computer verwendet. Im ersten System (S1) ist eine Nvidia GTX970 GPU, einen Intel i7 4950K CPU und 16GB Arbeitsspeicher installiert. Das zweite System besitzt eine Nvidia GTX1070 GPU, einen Intel i7 6700K 
CPU und ebenfalls 16GB Arbeitsspeicher. Die Unterschiede in der Hardware haben keinen Einfluss auf die Resultate der Experimente, da auf beiden Systemen dasselbe Betriebssystem mit den gleichen Softwarepaketen verwendet wird.
\fi