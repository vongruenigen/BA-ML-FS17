\chapter*{Abstract}
Das Ziel dieser Arbeit war ein Nachbau des Dialogsystems aus dem Paper \cite{}, inspiriert durch Beispiele maschineller Antworten wie: Human: "what is the purpose of being intelligent?" Machine: "to find out what it is .". Die Bedingungen für den Nachbaue waren ein wenig erschwert, da wir nu auf einer GPU die Berechnungen durchführten.

Während dem Nachbau mussten wir einige technische Herausforderungen meistern, bis wir dann endlich ein funktionsfähiges Modell hatten. Die Modelle wurden schlussendlich mit Tensorflow und Seq2seq2 umgesetzt. Wir trainierten anschliessen über 3 Wochen lang zwei Modelle mit unterschiedlichen Trainingsdate. Ein Modell trainierten wir mit OpenSubtitles welche gesprochene Sprache beinhaltet und Eines mit Reddit welcher geschriebener Sprache enthält. 

Anschliesssend untersuchten wir zuerst "quantitativ"? (oder so hast du das genannt) die Entwicklung der Modelle. Während die Trainings und Validierungsmetriken mit fortlaufendem Training besser wurden, wurden diejenigen der Testdaten eher schlechter, oder blieben gleich. Die ernüchternden Resultate waren insofern eine überraschung, da wir beim interagieren durchaus eine Verbesserung verspürten. Aufgrund dieses missmatch untersuchten wir die Vielfalt der Sprache. Diese Ergebnisse deckten sich mit unserem Subjektiven Empfinden.

Am Ende der Arbeit vergleichen wir Anhand einiger ausgesuchter Beispiele unsere replies mit denjenigen von Cleverbot. Zusätzlich entnahmen wir Dialoge aus dem paper \cites{} und verglichen diese mit unseren. Dabei kann unser Modell durchaus mithalten, hätte aber vermutlich in einem breit angelegten Test wenig Chance. Vor allem bei komplexeren Aussagen, scheint unser Modell schwierigkeiten zu haben. Als Ursache vermuten wir einerseits die Grösse des Modells und die Dauer des Trainings. Insgesamt sehen wir den Nachbau aber als Erfolg Aussagen wie: Human: "What are you?" Machine: "i m a bot .",(komma hier, keine ahnug) waren doch überraschend.



\chapter*{Abstract}
In der vorliegenden Arbeit werden die Möglichkeiten von Crossdomain Sentiment-Analyse mithilfe von Convolutional Neural Networks untersucht. Dabei ist das Ziel zu eruieren, inwiefern sich ein einzelner Sentiment-Klassifizierer auf verschiedenen Domänen verhält und ob es Sinn macht, Datensätze mehrerer Domänen in Kombination zu verwenden.

Als Erstes wird analysiert wie stark der Einfluss unterschiedlicher Word-Embeddings und Distant-Phasen auf einen Sentiment-Klassifizierer im Generellen ist und ob ein Zusammenhang zwischen den einzelnen Domänen und der besten Kombination an Word-Embeddings und Distant-Phasen existiert. Die Resultate zeigen, dass nicht anhand eines generellen Schemas entschieden werden kann welches die optimale Wahl ist. Dies muss im Einzelfall analysiert werden.

Im zweiten Teil wird untersucht, wie sich verschiedene Domänen im gegebenen Kontext verhalten. Dabei werden verschiedene Sentiment-Klassfizierer auf unterschiedlichen Domänen trainiert und dann auf allen anderen Domänen evaluiert. Die Resultate zeigen, dass es keine einzelnen Domänen gibt, welche sich besser eignet, um einen generalistischen Sentiment-Klassifizierer zu trainieren. Die Varianz der Resultate ist erstaunlich hoch, was nicht nur mit den unterschiedlichen Domänen, sondern auch mit den eingeführten Eigenschaften der Eindeutigkeit und Konzentration der Sentiments in Texten zusammenhängt. Die Generalisierung von einer Domäne zu einer anderen ist meistens nicht ohne weiteres möglich.

Im nächsten Teil wird untersucht, ob es sich lohnt, Daten aus unterschiedlichen Domänen für das Training eines Sentiment-Klassifizierers zu verwenden. Dafür werden sogenannte \quotes{Augmentation} Experimente durchgeführt: Bei diesen werden Daten aus mehreren Domänen in verschiedenen Verhältnissen durchmischt und danach wird der resultierende Sentiment-Klassifizierer auf einzelnen Domänen evaluiert. Die Resultate der Experimente zeigen, dass sich dieses Vorgehen von Vorteil ist, sofern zu wenige annotierte Daten der Zieldomäne zur Verfügung stehen.

Zuletzt wird untersucht, wie die Performanz eines generalistisch ausgelegten Sentiment-Klassifiziers auf einzelnen Domänen ist. Dafür wurden mehrere dieser Klassifizierer auf sogenannten \quotes{Ablation} Datensätzen trainiert. Dabei kann festgestellt werden, dass die Performanz eines generalistischen Sentiment-Klassifiziers auf einzelnen Domänen nicht bedeutend schlechter ist und für gewisse Anwendungsfälle durchaus eine Option darstellt.



In the following work we are going to investigate the potential of crossdomain sentiment-classification using convolutional neural networks. The main point of this research is to explore the topic of combining data from multiple sources to train such sentiment-classifiers.

In the first part, we analyse how well the sentiment-classifiers for different domains work together with multiple combinations of distant-phases and word-embeddings. It is shown, that there's no a priori answer on which combination is the best upfront; it has to be evaluated on each domain separately.

The next part focuses on how well specialized sentiment-classifiers work on different domains. For this purpose, multiple sentiment-classifiers are trained on a single domain and evaluated on all others. The results show, that there is no single best domain to train a generalized classifier and that the performance deteriorate if different domains are used for training and testing. Further, we determined that ambiguity in sentiments and the length of texts has an impact on the performance of the resulting classifier.

We conducted augmentation experiments to investigate the issue of combining data from multiple domains to train a sentiment-classifier. This means that different combinations of data from multiple domains is used to train the classifiers. The results of the experiments show, such an approach is only favorable if there is too little annotated data of the target domain.

As the last point, we evaluate the generalization performance of a sentiment-classifier. For this purpose, we used ablation datasets, which are combinations of all domains except for the target domain to train the classifiers. The results show that the performance of a generalized sentiment-classifier is only slightly worse than the one of specialized classifers and can such a procedure can even be useful in certain use-cases.
