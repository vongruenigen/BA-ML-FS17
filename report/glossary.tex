% Glossar entries

\newglossaryentry{cell}{name={cell},description={Any instance of any kind of RNN layer is called a \emph{cell}.}}

\newglossaryentry{epoch}
{
  name={Epoche},
  description={Ein Durchlauf durch alle Trainingsdaten während des Trainings eines CNN wird Epoche genannt}
}

\newglossaryentry{filter}
{
  name={Filter},
  description={Grundbausteine eines CNN. Werden verwendet um die Aktivierungen einer Convolutional Schicht zu berechnen}
}

\newglossaryentry{neuron}
{
  name={Neuron},
  description={Grundbausteine eines NN. Bilden eine mathematische Funktion ab}
}

\newglossaryentry{precision}
{
  name={Präzision},
  description={Wert, welcher angibt wieviel Prozent der beurteilten Datensätze korrekt klassifiziert wurden}
}

\newglossaryentry{layer}
{
  name={Schicht},
  description={Sammlung von mehreren Neuronen. Ein NN setzt sich im Normalfall aus mehreren solcher Schichten zusammen}
}

% Acronyms
\newacronym{GPU}{GPU}{Graphical Processing Unit}
\newacronym{NN}{NN}{Neural Network}
\newacronym{RNN}{RNN}{Recurrent Neural Network}
\newacronym{TBTT}{TBTT}{Truncated Backpropagation Through Time}
\newacronym{LSTM}{LSTM}{Long Short-Term Memory Network}

